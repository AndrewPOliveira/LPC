\documentclass[11pt]{article}

    \usepackage[breakable]{tcolorbox}
    \usepackage{parskip} % Stop auto-indenting (to mimic markdown behaviour)
    
    \usepackage{iftex}
    \ifPDFTeX
    	\usepackage[T1]{fontenc}
    	\usepackage{mathpazo}
    \else
    	\usepackage{fontspec}
    \fi

    % Basic figure setup, for now with no caption control since it's done
    % automatically by Pandoc (which extracts ![](path) syntax from Markdown).
    \usepackage{graphicx}
    % Maintain compatibility with old templates. Remove in nbconvert 6.0
    \let\Oldincludegraphics\includegraphics
    % Ensure that by default, figures have no caption (until we provide a
    % proper Figure object with a Caption API and a way to capture that
    % in the conversion process - todo).
    \usepackage{caption}
    \DeclareCaptionFormat{nocaption}{}
    \captionsetup{format=nocaption,aboveskip=0pt,belowskip=0pt}

    \usepackage[Export]{adjustbox} % Used to constrain images to a maximum size
    \adjustboxset{max size={0.9\linewidth}{0.9\paperheight}}
    \usepackage{float}
    \floatplacement{figure}{H} % forces figures to be placed at the correct location
    \usepackage{xcolor} % Allow colors to be defined
    \usepackage{enumerate} % Needed for markdown enumerations to work
    \usepackage{geometry} % Used to adjust the document margins
    \usepackage{amsmath} % Equations
    \usepackage{amssymb} % Equations
    \usepackage{textcomp} % defines textquotesingle
    % Hack from http://tex.stackexchange.com/a/47451/13684:
    \AtBeginDocument{%
        \def\PYZsq{\textquotesingle}% Upright quotes in Pygmentized code
    }
    \usepackage{upquote} % Upright quotes for verbatim code
    \usepackage{eurosym} % defines \euro
    \usepackage[mathletters]{ucs} % Extended unicode (utf-8) support
    \usepackage{fancyvrb} % verbatim replacement that allows latex
    \usepackage{grffile} % extends the file name processing of package graphics 
                         % to support a larger range
    \makeatletter % fix for grffile with XeLaTeX
    \def\Gread@@xetex#1{%
      \IfFileExists{"\Gin@base".bb}%
      {\Gread@eps{\Gin@base.bb}}%
      {\Gread@@xetex@aux#1}%
    }
    \makeatother

    % The hyperref package gives us a pdf with properly built
    % internal navigation ('pdf bookmarks' for the table of contents,
    % internal cross-reference links, web links for URLs, etc.)
    \usepackage{hyperref}
    % The default LaTeX title has an obnoxious amount of whitespace. By default,
    % titling removes some of it. It also provides customization options.
    \usepackage{titling}
    \usepackage{longtable} % longtable support required by pandoc >1.10
    \usepackage{booktabs}  % table support for pandoc > 1.12.2
    \usepackage[inline]{enumitem} % IRkernel/repr support (it uses the enumerate* environment)
    \usepackage[normalem]{ulem} % ulem is needed to support strikethroughs (\sout)
                                % normalem makes italics be italics, not underlines
    \usepackage{mathrsfs}
    

    
    % Colors for the hyperref package
    \definecolor{urlcolor}{rgb}{0,.145,.698}
    \definecolor{linkcolor}{rgb}{.71,0.21,0.01}
    \definecolor{citecolor}{rgb}{.12,.54,.11}

    % ANSI colors
    \definecolor{ansi-black}{HTML}{3E424D}
    \definecolor{ansi-black-intense}{HTML}{282C36}
    \definecolor{ansi-red}{HTML}{E75C58}
    \definecolor{ansi-red-intense}{HTML}{B22B31}
    \definecolor{ansi-green}{HTML}{00A250}
    \definecolor{ansi-green-intense}{HTML}{007427}
    \definecolor{ansi-yellow}{HTML}{DDB62B}
    \definecolor{ansi-yellow-intense}{HTML}{B27D12}
    \definecolor{ansi-blue}{HTML}{208FFB}
    \definecolor{ansi-blue-intense}{HTML}{0065CA}
    \definecolor{ansi-magenta}{HTML}{D160C4}
    \definecolor{ansi-magenta-intense}{HTML}{A03196}
    \definecolor{ansi-cyan}{HTML}{60C6C8}
    \definecolor{ansi-cyan-intense}{HTML}{258F8F}
    \definecolor{ansi-white}{HTML}{C5C1B4}
    \definecolor{ansi-white-intense}{HTML}{A1A6B2}
    \definecolor{ansi-default-inverse-fg}{HTML}{FFFFFF}
    \definecolor{ansi-default-inverse-bg}{HTML}{000000}

    % commands and environments needed by pandoc snippets
    % extracted from the output of `pandoc -s`
    \providecommand{\tightlist}{%
      \setlength{\itemsep}{0pt}\setlength{\parskip}{0pt}}
    \DefineVerbatimEnvironment{Highlighting}{Verbatim}{commandchars=\\\{\}}
    % Add ',fontsize=\small' for more characters per line
    \newenvironment{Shaded}{}{}
    \newcommand{\KeywordTok}[1]{\textcolor[rgb]{0.00,0.44,0.13}{\textbf{{#1}}}}
    \newcommand{\DataTypeTok}[1]{\textcolor[rgb]{0.56,0.13,0.00}{{#1}}}
    \newcommand{\DecValTok}[1]{\textcolor[rgb]{0.25,0.63,0.44}{{#1}}}
    \newcommand{\BaseNTok}[1]{\textcolor[rgb]{0.25,0.63,0.44}{{#1}}}
    \newcommand{\FloatTok}[1]{\textcolor[rgb]{0.25,0.63,0.44}{{#1}}}
    \newcommand{\CharTok}[1]{\textcolor[rgb]{0.25,0.44,0.63}{{#1}}}
    \newcommand{\StringTok}[1]{\textcolor[rgb]{0.25,0.44,0.63}{{#1}}}
    \newcommand{\CommentTok}[1]{\textcolor[rgb]{0.38,0.63,0.69}{\textit{{#1}}}}
    \newcommand{\OtherTok}[1]{\textcolor[rgb]{0.00,0.44,0.13}{{#1}}}
    \newcommand{\AlertTok}[1]{\textcolor[rgb]{1.00,0.00,0.00}{\textbf{{#1}}}}
    \newcommand{\FunctionTok}[1]{\textcolor[rgb]{0.02,0.16,0.49}{{#1}}}
    \newcommand{\RegionMarkerTok}[1]{{#1}}
    \newcommand{\ErrorTok}[1]{\textcolor[rgb]{1.00,0.00,0.00}{\textbf{{#1}}}}
    \newcommand{\NormalTok}[1]{{#1}}
    
    % Additional commands for more recent versions of Pandoc
    \newcommand{\ConstantTok}[1]{\textcolor[rgb]{0.53,0.00,0.00}{{#1}}}
    \newcommand{\SpecialCharTok}[1]{\textcolor[rgb]{0.25,0.44,0.63}{{#1}}}
    \newcommand{\VerbatimStringTok}[1]{\textcolor[rgb]{0.25,0.44,0.63}{{#1}}}
    \newcommand{\SpecialStringTok}[1]{\textcolor[rgb]{0.73,0.40,0.53}{{#1}}}
    \newcommand{\ImportTok}[1]{{#1}}
    \newcommand{\DocumentationTok}[1]{\textcolor[rgb]{0.73,0.13,0.13}{\textit{{#1}}}}
    \newcommand{\AnnotationTok}[1]{\textcolor[rgb]{0.38,0.63,0.69}{\textbf{\textit{{#1}}}}}
    \newcommand{\CommentVarTok}[1]{\textcolor[rgb]{0.38,0.63,0.69}{\textbf{\textit{{#1}}}}}
    \newcommand{\VariableTok}[1]{\textcolor[rgb]{0.10,0.09,0.49}{{#1}}}
    \newcommand{\ControlFlowTok}[1]{\textcolor[rgb]{0.00,0.44,0.13}{\textbf{{#1}}}}
    \newcommand{\OperatorTok}[1]{\textcolor[rgb]{0.40,0.40,0.40}{{#1}}}
    \newcommand{\BuiltInTok}[1]{{#1}}
    \newcommand{\ExtensionTok}[1]{{#1}}
    \newcommand{\PreprocessorTok}[1]{\textcolor[rgb]{0.74,0.48,0.00}{{#1}}}
    \newcommand{\AttributeTok}[1]{\textcolor[rgb]{0.49,0.56,0.16}{{#1}}}
    \newcommand{\InformationTok}[1]{\textcolor[rgb]{0.38,0.63,0.69}{\textbf{\textit{{#1}}}}}
    \newcommand{\WarningTok}[1]{\textcolor[rgb]{0.38,0.63,0.69}{\textbf{\textit{{#1}}}}}
    
    
    % Define a nice break command that doesn't care if a line doesn't already
    % exist.
    \def\br{\hspace*{\fill} \\* }
    % Math Jax compatibility definitions
    \def\gt{>}
    \def\lt{<}
    \let\Oldtex\TeX
    \let\Oldlatex\LaTeX
    \renewcommand{\TeX}{\textrm{\Oldtex}}
    \renewcommand{\LaTeX}{\textrm{\Oldlatex}}
    % Document parameters
    % Document title
    \title{Lab1}
    
    
    
    
    
% Pygments definitions
\makeatletter
\def\PY@reset{\let\PY@it=\relax \let\PY@bf=\relax%
    \let\PY@ul=\relax \let\PY@tc=\relax%
    \let\PY@bc=\relax \let\PY@ff=\relax}
\def\PY@tok#1{\csname PY@tok@#1\endcsname}
\def\PY@toks#1+{\ifx\relax#1\empty\else%
    \PY@tok{#1}\expandafter\PY@toks\fi}
\def\PY@do#1{\PY@bc{\PY@tc{\PY@ul{%
    \PY@it{\PY@bf{\PY@ff{#1}}}}}}}
\def\PY#1#2{\PY@reset\PY@toks#1+\relax+\PY@do{#2}}

\expandafter\def\csname PY@tok@w\endcsname{\def\PY@tc##1{\textcolor[rgb]{0.73,0.73,0.73}{##1}}}
\expandafter\def\csname PY@tok@c\endcsname{\let\PY@it=\textit\def\PY@tc##1{\textcolor[rgb]{0.25,0.50,0.50}{##1}}}
\expandafter\def\csname PY@tok@cp\endcsname{\def\PY@tc##1{\textcolor[rgb]{0.74,0.48,0.00}{##1}}}
\expandafter\def\csname PY@tok@k\endcsname{\let\PY@bf=\textbf\def\PY@tc##1{\textcolor[rgb]{0.00,0.50,0.00}{##1}}}
\expandafter\def\csname PY@tok@kp\endcsname{\def\PY@tc##1{\textcolor[rgb]{0.00,0.50,0.00}{##1}}}
\expandafter\def\csname PY@tok@kt\endcsname{\def\PY@tc##1{\textcolor[rgb]{0.69,0.00,0.25}{##1}}}
\expandafter\def\csname PY@tok@o\endcsname{\def\PY@tc##1{\textcolor[rgb]{0.40,0.40,0.40}{##1}}}
\expandafter\def\csname PY@tok@ow\endcsname{\let\PY@bf=\textbf\def\PY@tc##1{\textcolor[rgb]{0.67,0.13,1.00}{##1}}}
\expandafter\def\csname PY@tok@nb\endcsname{\def\PY@tc##1{\textcolor[rgb]{0.00,0.50,0.00}{##1}}}
\expandafter\def\csname PY@tok@nf\endcsname{\def\PY@tc##1{\textcolor[rgb]{0.00,0.00,1.00}{##1}}}
\expandafter\def\csname PY@tok@nc\endcsname{\let\PY@bf=\textbf\def\PY@tc##1{\textcolor[rgb]{0.00,0.00,1.00}{##1}}}
\expandafter\def\csname PY@tok@nn\endcsname{\let\PY@bf=\textbf\def\PY@tc##1{\textcolor[rgb]{0.00,0.00,1.00}{##1}}}
\expandafter\def\csname PY@tok@ne\endcsname{\let\PY@bf=\textbf\def\PY@tc##1{\textcolor[rgb]{0.82,0.25,0.23}{##1}}}
\expandafter\def\csname PY@tok@nv\endcsname{\def\PY@tc##1{\textcolor[rgb]{0.10,0.09,0.49}{##1}}}
\expandafter\def\csname PY@tok@no\endcsname{\def\PY@tc##1{\textcolor[rgb]{0.53,0.00,0.00}{##1}}}
\expandafter\def\csname PY@tok@nl\endcsname{\def\PY@tc##1{\textcolor[rgb]{0.63,0.63,0.00}{##1}}}
\expandafter\def\csname PY@tok@ni\endcsname{\let\PY@bf=\textbf\def\PY@tc##1{\textcolor[rgb]{0.60,0.60,0.60}{##1}}}
\expandafter\def\csname PY@tok@na\endcsname{\def\PY@tc##1{\textcolor[rgb]{0.49,0.56,0.16}{##1}}}
\expandafter\def\csname PY@tok@nt\endcsname{\let\PY@bf=\textbf\def\PY@tc##1{\textcolor[rgb]{0.00,0.50,0.00}{##1}}}
\expandafter\def\csname PY@tok@nd\endcsname{\def\PY@tc##1{\textcolor[rgb]{0.67,0.13,1.00}{##1}}}
\expandafter\def\csname PY@tok@s\endcsname{\def\PY@tc##1{\textcolor[rgb]{0.73,0.13,0.13}{##1}}}
\expandafter\def\csname PY@tok@sd\endcsname{\let\PY@it=\textit\def\PY@tc##1{\textcolor[rgb]{0.73,0.13,0.13}{##1}}}
\expandafter\def\csname PY@tok@si\endcsname{\let\PY@bf=\textbf\def\PY@tc##1{\textcolor[rgb]{0.73,0.40,0.53}{##1}}}
\expandafter\def\csname PY@tok@se\endcsname{\let\PY@bf=\textbf\def\PY@tc##1{\textcolor[rgb]{0.73,0.40,0.13}{##1}}}
\expandafter\def\csname PY@tok@sr\endcsname{\def\PY@tc##1{\textcolor[rgb]{0.73,0.40,0.53}{##1}}}
\expandafter\def\csname PY@tok@ss\endcsname{\def\PY@tc##1{\textcolor[rgb]{0.10,0.09,0.49}{##1}}}
\expandafter\def\csname PY@tok@sx\endcsname{\def\PY@tc##1{\textcolor[rgb]{0.00,0.50,0.00}{##1}}}
\expandafter\def\csname PY@tok@m\endcsname{\def\PY@tc##1{\textcolor[rgb]{0.40,0.40,0.40}{##1}}}
\expandafter\def\csname PY@tok@gh\endcsname{\let\PY@bf=\textbf\def\PY@tc##1{\textcolor[rgb]{0.00,0.00,0.50}{##1}}}
\expandafter\def\csname PY@tok@gu\endcsname{\let\PY@bf=\textbf\def\PY@tc##1{\textcolor[rgb]{0.50,0.00,0.50}{##1}}}
\expandafter\def\csname PY@tok@gd\endcsname{\def\PY@tc##1{\textcolor[rgb]{0.63,0.00,0.00}{##1}}}
\expandafter\def\csname PY@tok@gi\endcsname{\def\PY@tc##1{\textcolor[rgb]{0.00,0.63,0.00}{##1}}}
\expandafter\def\csname PY@tok@gr\endcsname{\def\PY@tc##1{\textcolor[rgb]{1.00,0.00,0.00}{##1}}}
\expandafter\def\csname PY@tok@ge\endcsname{\let\PY@it=\textit}
\expandafter\def\csname PY@tok@gs\endcsname{\let\PY@bf=\textbf}
\expandafter\def\csname PY@tok@gp\endcsname{\let\PY@bf=\textbf\def\PY@tc##1{\textcolor[rgb]{0.00,0.00,0.50}{##1}}}
\expandafter\def\csname PY@tok@go\endcsname{\def\PY@tc##1{\textcolor[rgb]{0.53,0.53,0.53}{##1}}}
\expandafter\def\csname PY@tok@gt\endcsname{\def\PY@tc##1{\textcolor[rgb]{0.00,0.27,0.87}{##1}}}
\expandafter\def\csname PY@tok@err\endcsname{\def\PY@bc##1{\setlength{\fboxsep}{0pt}\fcolorbox[rgb]{1.00,0.00,0.00}{1,1,1}{\strut ##1}}}
\expandafter\def\csname PY@tok@kc\endcsname{\let\PY@bf=\textbf\def\PY@tc##1{\textcolor[rgb]{0.00,0.50,0.00}{##1}}}
\expandafter\def\csname PY@tok@kd\endcsname{\let\PY@bf=\textbf\def\PY@tc##1{\textcolor[rgb]{0.00,0.50,0.00}{##1}}}
\expandafter\def\csname PY@tok@kn\endcsname{\let\PY@bf=\textbf\def\PY@tc##1{\textcolor[rgb]{0.00,0.50,0.00}{##1}}}
\expandafter\def\csname PY@tok@kr\endcsname{\let\PY@bf=\textbf\def\PY@tc##1{\textcolor[rgb]{0.00,0.50,0.00}{##1}}}
\expandafter\def\csname PY@tok@bp\endcsname{\def\PY@tc##1{\textcolor[rgb]{0.00,0.50,0.00}{##1}}}
\expandafter\def\csname PY@tok@fm\endcsname{\def\PY@tc##1{\textcolor[rgb]{0.00,0.00,1.00}{##1}}}
\expandafter\def\csname PY@tok@vc\endcsname{\def\PY@tc##1{\textcolor[rgb]{0.10,0.09,0.49}{##1}}}
\expandafter\def\csname PY@tok@vg\endcsname{\def\PY@tc##1{\textcolor[rgb]{0.10,0.09,0.49}{##1}}}
\expandafter\def\csname PY@tok@vi\endcsname{\def\PY@tc##1{\textcolor[rgb]{0.10,0.09,0.49}{##1}}}
\expandafter\def\csname PY@tok@vm\endcsname{\def\PY@tc##1{\textcolor[rgb]{0.10,0.09,0.49}{##1}}}
\expandafter\def\csname PY@tok@sa\endcsname{\def\PY@tc##1{\textcolor[rgb]{0.73,0.13,0.13}{##1}}}
\expandafter\def\csname PY@tok@sb\endcsname{\def\PY@tc##1{\textcolor[rgb]{0.73,0.13,0.13}{##1}}}
\expandafter\def\csname PY@tok@sc\endcsname{\def\PY@tc##1{\textcolor[rgb]{0.73,0.13,0.13}{##1}}}
\expandafter\def\csname PY@tok@dl\endcsname{\def\PY@tc##1{\textcolor[rgb]{0.73,0.13,0.13}{##1}}}
\expandafter\def\csname PY@tok@s2\endcsname{\def\PY@tc##1{\textcolor[rgb]{0.73,0.13,0.13}{##1}}}
\expandafter\def\csname PY@tok@sh\endcsname{\def\PY@tc##1{\textcolor[rgb]{0.73,0.13,0.13}{##1}}}
\expandafter\def\csname PY@tok@s1\endcsname{\def\PY@tc##1{\textcolor[rgb]{0.73,0.13,0.13}{##1}}}
\expandafter\def\csname PY@tok@mb\endcsname{\def\PY@tc##1{\textcolor[rgb]{0.40,0.40,0.40}{##1}}}
\expandafter\def\csname PY@tok@mf\endcsname{\def\PY@tc##1{\textcolor[rgb]{0.40,0.40,0.40}{##1}}}
\expandafter\def\csname PY@tok@mh\endcsname{\def\PY@tc##1{\textcolor[rgb]{0.40,0.40,0.40}{##1}}}
\expandafter\def\csname PY@tok@mi\endcsname{\def\PY@tc##1{\textcolor[rgb]{0.40,0.40,0.40}{##1}}}
\expandafter\def\csname PY@tok@il\endcsname{\def\PY@tc##1{\textcolor[rgb]{0.40,0.40,0.40}{##1}}}
\expandafter\def\csname PY@tok@mo\endcsname{\def\PY@tc##1{\textcolor[rgb]{0.40,0.40,0.40}{##1}}}
\expandafter\def\csname PY@tok@ch\endcsname{\let\PY@it=\textit\def\PY@tc##1{\textcolor[rgb]{0.25,0.50,0.50}{##1}}}
\expandafter\def\csname PY@tok@cm\endcsname{\let\PY@it=\textit\def\PY@tc##1{\textcolor[rgb]{0.25,0.50,0.50}{##1}}}
\expandafter\def\csname PY@tok@cpf\endcsname{\let\PY@it=\textit\def\PY@tc##1{\textcolor[rgb]{0.25,0.50,0.50}{##1}}}
\expandafter\def\csname PY@tok@c1\endcsname{\let\PY@it=\textit\def\PY@tc##1{\textcolor[rgb]{0.25,0.50,0.50}{##1}}}
\expandafter\def\csname PY@tok@cs\endcsname{\let\PY@it=\textit\def\PY@tc##1{\textcolor[rgb]{0.25,0.50,0.50}{##1}}}

\def\PYZbs{\char`\\}
\def\PYZus{\char`\_}
\def\PYZob{\char`\{}
\def\PYZcb{\char`\}}
\def\PYZca{\char`\^}
\def\PYZam{\char`\&}
\def\PYZlt{\char`\<}
\def\PYZgt{\char`\>}
\def\PYZsh{\char`\#}
\def\PYZpc{\char`\%}
\def\PYZdl{\char`\$}
\def\PYZhy{\char`\-}
\def\PYZsq{\char`\'}
\def\PYZdq{\char`\"}
\def\PYZti{\char`\~}
% for compatibility with earlier versions
\def\PYZat{@}
\def\PYZlb{[}
\def\PYZrb{]}
\makeatother


    % For linebreaks inside Verbatim environment from package fancyvrb. 
    \makeatletter
        \newbox\Wrappedcontinuationbox 
        \newbox\Wrappedvisiblespacebox 
        \newcommand*\Wrappedvisiblespace {\textcolor{red}{\textvisiblespace}} 
        \newcommand*\Wrappedcontinuationsymbol {\textcolor{red}{\llap{\tiny$\m@th\hookrightarrow$}}} 
        \newcommand*\Wrappedcontinuationindent {3ex } 
        \newcommand*\Wrappedafterbreak {\kern\Wrappedcontinuationindent\copy\Wrappedcontinuationbox} 
        % Take advantage of the already applied Pygments mark-up to insert 
        % potential linebreaks for TeX processing. 
        %        {, <, #, %, $, ' and ": go to next line. 
        %        _, }, ^, &, >, - and ~: stay at end of broken line. 
        % Use of \textquotesingle for straight quote. 
        \newcommand*\Wrappedbreaksatspecials {% 
            \def\PYGZus{\discretionary{\char`\_}{\Wrappedafterbreak}{\char`\_}}% 
            \def\PYGZob{\discretionary{}{\Wrappedafterbreak\char`\{}{\char`\{}}% 
            \def\PYGZcb{\discretionary{\char`\}}{\Wrappedafterbreak}{\char`\}}}% 
            \def\PYGZca{\discretionary{\char`\^}{\Wrappedafterbreak}{\char`\^}}% 
            \def\PYGZam{\discretionary{\char`\&}{\Wrappedafterbreak}{\char`\&}}% 
            \def\PYGZlt{\discretionary{}{\Wrappedafterbreak\char`\<}{\char`\<}}% 
            \def\PYGZgt{\discretionary{\char`\>}{\Wrappedafterbreak}{\char`\>}}% 
            \def\PYGZsh{\discretionary{}{\Wrappedafterbreak\char`\#}{\char`\#}}% 
            \def\PYGZpc{\discretionary{}{\Wrappedafterbreak\char`\%}{\char`\%}}% 
            \def\PYGZdl{\discretionary{}{\Wrappedafterbreak\char`\$}{\char`\$}}% 
            \def\PYGZhy{\discretionary{\char`\-}{\Wrappedafterbreak}{\char`\-}}% 
            \def\PYGZsq{\discretionary{}{\Wrappedafterbreak\textquotesingle}{\textquotesingle}}% 
            \def\PYGZdq{\discretionary{}{\Wrappedafterbreak\char`\"}{\char`\"}}% 
            \def\PYGZti{\discretionary{\char`\~}{\Wrappedafterbreak}{\char`\~}}% 
        } 
        % Some characters . , ; ? ! / are not pygmentized. 
        % This macro makes them "active" and they will insert potential linebreaks 
        \newcommand*\Wrappedbreaksatpunct {% 
            \lccode`\~`\.\lowercase{\def~}{\discretionary{\hbox{\char`\.}}{\Wrappedafterbreak}{\hbox{\char`\.}}}% 
            \lccode`\~`\,\lowercase{\def~}{\discretionary{\hbox{\char`\,}}{\Wrappedafterbreak}{\hbox{\char`\,}}}% 
            \lccode`\~`\;\lowercase{\def~}{\discretionary{\hbox{\char`\;}}{\Wrappedafterbreak}{\hbox{\char`\;}}}% 
            \lccode`\~`\:\lowercase{\def~}{\discretionary{\hbox{\char`\:}}{\Wrappedafterbreak}{\hbox{\char`\:}}}% 
            \lccode`\~`\?\lowercase{\def~}{\discretionary{\hbox{\char`\?}}{\Wrappedafterbreak}{\hbox{\char`\?}}}% 
            \lccode`\~`\!\lowercase{\def~}{\discretionary{\hbox{\char`\!}}{\Wrappedafterbreak}{\hbox{\char`\!}}}% 
            \lccode`\~`\/\lowercase{\def~}{\discretionary{\hbox{\char`\/}}{\Wrappedafterbreak}{\hbox{\char`\/}}}% 
            \catcode`\.\active
            \catcode`\,\active 
            \catcode`\;\active
            \catcode`\:\active
            \catcode`\?\active
            \catcode`\!\active
            \catcode`\/\active 
            \lccode`\~`\~ 	
        }
    \makeatother

    \let\OriginalVerbatim=\Verbatim
    \makeatletter
    \renewcommand{\Verbatim}[1][1]{%
        %\parskip\z@skip
        \sbox\Wrappedcontinuationbox {\Wrappedcontinuationsymbol}%
        \sbox\Wrappedvisiblespacebox {\FV@SetupFont\Wrappedvisiblespace}%
        \def\FancyVerbFormatLine ##1{\hsize\linewidth
            \vtop{\raggedright\hyphenpenalty\z@\exhyphenpenalty\z@
                \doublehyphendemerits\z@\finalhyphendemerits\z@
                \strut ##1\strut}%
        }%
        % If the linebreak is at a space, the latter will be displayed as visible
        % space at end of first line, and a continuation symbol starts next line.
        % Stretch/shrink are however usually zero for typewriter font.
        \def\FV@Space {%
            \nobreak\hskip\z@ plus\fontdimen3\font minus\fontdimen4\font
            \discretionary{\copy\Wrappedvisiblespacebox}{\Wrappedafterbreak}
            {\kern\fontdimen2\font}%
        }%
        
        % Allow breaks at special characters using \PYG... macros.
        \Wrappedbreaksatspecials
        % Breaks at punctuation characters . , ; ? ! and / need catcode=\active 	
        \OriginalVerbatim[#1,codes*=\Wrappedbreaksatpunct]%
    }
    \makeatother

    % Exact colors from NB
    \definecolor{incolor}{HTML}{303F9F}
    \definecolor{outcolor}{HTML}{D84315}
    \definecolor{cellborder}{HTML}{CFCFCF}
    \definecolor{cellbackground}{HTML}{F7F7F7}
    
    % prompt
    \makeatletter
    \newcommand{\boxspacing}{\kern\kvtcb@left@rule\kern\kvtcb@boxsep}
    \makeatother
    \newcommand{\prompt}[4]{
        \ttfamily\llap{{\color{#2}[#3]:\hspace{3pt}#4}}\vspace{-\baselineskip}
    }
    

    
    % Prevent overflowing lines due to hard-to-break entities
    \sloppy 
    % Setup hyperref package
    \hypersetup{
      breaklinks=true,  % so long urls are correctly broken across lines
      colorlinks=true,
      urlcolor=urlcolor,
      linkcolor=linkcolor,
      citecolor=citecolor,
      }
    % Slightly bigger margins than the latex defaults
    
    \geometry{verbose,tmargin=1in,bmargin=1in,lmargin=1in,rmargin=1in}
    
    

\begin{document}
    
    \maketitle
    
    

    
    \begin{tcolorbox}[breakable, size=fbox, boxrule=1pt, pad at break*=1mm,colback=cellbackground, colframe=cellborder]
\prompt{In}{incolor}{167}{\boxspacing}
\begin{Verbatim}[commandchars=\\\{\}]
\PY{k+kn}{import} \PY{n+nn}{matplotlib}\PY{n+nn}{.}\PY{n+nn}{pyplot} \PY{k}{as} \PY{n+nn}{plt}
\PY{k+kn}{import} \PY{n+nn}{numpy} \PY{k}{as} \PY{n+nn}{np}
\PY{k+kn}{from} \PY{n+nn}{numpy} \PY{k}{import} \PY{n}{cos}\PY{p}{,} \PY{n}{sin}\PY{p}{,} \PY{n}{arctan2}\PY{p}{,} \PY{n}{sqrt}
\end{Verbatim}
\end{tcolorbox}

    \begin{tcolorbox}[breakable, size=fbox, boxrule=1pt, pad at break*=1mm,colback=cellbackground, colframe=cellborder]
\prompt{In}{incolor}{168}{\boxspacing}
\begin{Verbatim}[commandchars=\\\{\}]
\PY{k+kn}{from} \PY{n+nn}{IPython}\PY{n+nn}{.}\PY{n+nn}{core}\PY{n+nn}{.}\PY{n+nn}{pylabtools} \PY{k}{import} \PY{n}{figsize}
\PY{n}{figsize}\PY{p}{(}\PY{l+m+mi}{9}\PY{p}{,} \PY{l+m+mi}{3}\PY{p}{)}

\PY{k+kn}{from} \PY{n+nn}{IPython}\PY{n+nn}{.}\PY{n+nn}{core}\PY{n+nn}{.}\PY{n+nn}{display} \PY{k}{import} \PY{n}{HTML}
\PY{n}{HTML}\PY{p}{(}\PY{l+s+s2}{\PYZdq{}\PYZdq{}\PYZdq{}}
\PY{l+s+s2}{\PYZlt{}style\PYZgt{}}
\PY{l+s+s2}{.output\PYZus{}png }\PY{l+s+s2}{\PYZob{}}
\PY{l+s+s2}{    display: table\PYZhy{}cell;}
\PY{l+s+s2}{    text\PYZhy{}align: center;}
\PY{l+s+s2}{    vertical\PYZhy{}align: middle;}
\PY{l+s+s2}{\PYZcb{}}
\PY{l+s+s2}{\PYZlt{}/style\PYZgt{}}
\PY{l+s+s2}{\PYZdq{}\PYZdq{}\PYZdq{}}\PY{p}{)}
\end{Verbatim}
\end{tcolorbox}

            \begin{tcolorbox}[breakable, size=fbox, boxrule=.5pt, pad at break*=1mm, opacityfill=0]
\prompt{Out}{outcolor}{168}{\boxspacing}
\begin{Verbatim}[commandchars=\\\{\}]
<IPython.core.display.HTML object>
\end{Verbatim}
\end{tcolorbox}
        
    \hypertarget{representauxe7uxe3o-de-funuxe7uxf5es-periuxf3dicas-via-suxe9ries-de-fourier}{%
\subsubsection{Representação de funções periódicas via séries de
Fourier}\label{representauxe7uxe3o-de-funuxe7uxf5es-periuxf3dicas-via-suxe9ries-de-fourier}}

Seja a função \(f(t)\) periódica com período \(P\), isto é
\(f(t)=f(t+nP)\) para \(n=1,2,3,...\), e absolutamente integrável num
intervalo de comprimento \(P\). Considerando que a função \(f(t)\)
admite representação em termos de uma série trigonométrica de Fourier,
\(f(t)\) pode ser escrita como

\[ f(t)=\frac{a_{0}}{2}+\sum_{n=1}^{\infty}\left[a_{n} \cos \left(\frac{n \pi t}{T}\right)+b_{n} \sin \left(\frac{n \pi t}{T}\right)\right] \]

em que \(a_n\) e \(b_n\) são os coeficientes de Fourier da expansão de
\(f(t)\), calculados de acordo com as expressões

\[ a_{n}=\frac{2}{P} \int_{P} f(t) \cos \left(2\pi t\frac{n}{P}\right) dt \\
b_{n}=\frac{2}{P} \int_{P} f(t) \sin \left(2\pi t\frac{n}{P}\right) dt.\\ \]

Em particular, temos que 

\begin{eqnarray}
a_{0}&=&\frac{2}{P} \int_{P} f(t) \cos(0) dt =\frac{2}{P} \int_{P} f(t) dt\\
b_{0}&=&\frac{2}{P} \int_{P} f(t) \sin(0) dt = 0.
\end{eqnarray}

    \begin{tcolorbox}[breakable, size=fbox, boxrule=1pt, pad at break*=1mm,colback=cellbackground, colframe=cellborder]
\prompt{In}{incolor}{169}{\boxspacing}
\begin{Verbatim}[commandchars=\\\{\}]
\PY{k+kn}{from} \PY{n+nn}{scipy}\PY{n+nn}{.}\PY{n+nn}{signal} \PY{k}{import} \PY{n}{square}\PY{p}{,} \PY{n}{sawtooth}

\PY{n}{f0} \PY{o}{=} \PY{l+m+mi}{100}     \PY{c+c1}{\PYZsh{} Frequência fundamental}
\PY{n}{fa} \PY{o}{=} \PY{l+m+mi}{200}\PY{o}{*}\PY{n}{f0}   \PY{c+c1}{\PYZsh{} Frequência de amostragem}

\PY{n}{t}    \PY{o}{=} \PY{n}{np}\PY{o}{.}\PY{n}{arange}\PY{p}{(}\PY{l+m+mi}{0}\PY{p}{,} \PY{l+m+mf}{40e\PYZhy{}3}\PY{p}{,} \PY{l+m+mi}{1}\PY{o}{/}\PY{n}{fa}\PY{p}{)}
\PY{n}{π}    \PY{o}{=} \PY{n}{np}\PY{o}{.}\PY{n}{pi}
\PY{n}{ϕ}    \PY{o}{=} \PY{n}{π}\PY{o}{/}\PY{l+m+mi}{6}

\PY{n}{quad}   \PY{o}{=} \PY{n}{square}\PY{p}{(}\PY{l+m+mi}{2}\PY{o}{*}\PY{n}{π}\PY{o}{*}\PY{n}{f0}\PY{o}{*}\PY{n}{t} \PY{o}{+} \PY{n}{ϕ}\PY{p}{)}                \PY{c+c1}{\PYZsh{} onda quadrada com frequência fundamental f0 e phase inicial ϕ}
\PY{n}{triang} \PY{o}{=} \PY{n}{sawtooth}\PY{p}{(}\PY{l+m+mi}{2}\PY{o}{*}\PY{n}{π}\PY{o}{*}\PY{n}{f0}\PY{o}{*}\PY{n}{t} \PY{o}{+} \PY{n}{ϕ}\PY{p}{,} \PY{n}{width} \PY{o}{=} \PY{l+m+mf}{0.5}\PY{p}{)} \PY{c+c1}{\PYZsh{} onda triangular com frequência fundamental f0 e phase inicial ϕ}
\PY{n}{dente}  \PY{o}{=} \PY{n}{sawtooth}\PY{p}{(}\PY{l+m+mi}{2}\PY{o}{*}\PY{n}{π}\PY{o}{*}\PY{n}{f0}\PY{o}{*}\PY{n}{t} \PY{o}{+} \PY{n}{ϕ}\PY{p}{)}              \PY{c+c1}{\PYZsh{} onda dente de serra com frequência fundamental f0 e phase inicial ϕ}

\PY{n}{plt}\PY{o}{.}\PY{n}{figure}\PY{p}{(}\PY{p}{)}
\PY{n}{plt}\PY{o}{.}\PY{n}{plot}\PY{p}{(}\PY{n}{t}\PY{p}{,} \PY{n}{quad}\PY{p}{,}\PY{l+s+s1}{\PYZsq{}}\PY{l+s+s1}{b}\PY{l+s+s1}{\PYZsq{}}\PY{p}{)}
\PY{n}{plt}\PY{o}{.}\PY{n}{ylim}\PY{p}{(}\PY{n}{quad}\PY{o}{.}\PY{n}{min}\PY{p}{(}\PY{l+m+mi}{0}\PY{p}{)}\PY{o}{\PYZhy{}}\PY{l+m+mf}{0.1}\PY{p}{,} \PY{n}{quad}\PY{o}{.}\PY{n}{max}\PY{p}{(}\PY{l+m+mi}{0}\PY{p}{)}\PY{o}{+}\PY{l+m+mf}{0.1}\PY{p}{)}
\PY{n}{plt}\PY{o}{.}\PY{n}{xlim}\PY{p}{(}\PY{l+m+mi}{0}\PY{p}{,}\PY{n}{t}\PY{o}{.}\PY{n}{max}\PY{p}{(}\PY{l+m+mi}{0}\PY{p}{)}\PY{p}{)}
\PY{n}{plt}\PY{o}{.}\PY{n}{xlabel}\PY{p}{(}\PY{l+s+s1}{\PYZsq{}}\PY{l+s+s1}{tempo (s)}\PY{l+s+s1}{\PYZsq{}}\PY{p}{)}
\PY{n}{plt}\PY{o}{.}\PY{n}{ylabel}\PY{p}{(}\PY{l+s+s1}{\PYZsq{}}\PY{l+s+s1}{amplitude}\PY{l+s+s1}{\PYZsq{}}\PY{p}{)}
\PY{n}{plt}\PY{o}{.}\PY{n}{grid}\PY{p}{(}\PY{p}{)}

\PY{n}{plt}\PY{o}{.}\PY{n}{figure}\PY{p}{(}\PY{p}{)}
\PY{n}{plt}\PY{o}{.}\PY{n}{plot}\PY{p}{(}\PY{n}{t}\PY{p}{,} \PY{n}{triang}\PY{p}{,}\PY{l+s+s1}{\PYZsq{}}\PY{l+s+s1}{b}\PY{l+s+s1}{\PYZsq{}}\PY{p}{)}
\PY{n}{plt}\PY{o}{.}\PY{n}{ylim}\PY{p}{(}\PY{n}{triang}\PY{o}{.}\PY{n}{min}\PY{p}{(}\PY{l+m+mi}{0}\PY{p}{)}\PY{o}{\PYZhy{}}\PY{l+m+mf}{0.1}\PY{p}{,} \PY{n}{triang}\PY{o}{.}\PY{n}{max}\PY{p}{(}\PY{l+m+mi}{0}\PY{p}{)}\PY{o}{+}\PY{l+m+mf}{0.1}\PY{p}{)}
\PY{n}{plt}\PY{o}{.}\PY{n}{xlim}\PY{p}{(}\PY{l+m+mi}{0}\PY{p}{,}\PY{n}{t}\PY{o}{.}\PY{n}{max}\PY{p}{(}\PY{l+m+mi}{0}\PY{p}{)}\PY{p}{)}
\PY{n}{plt}\PY{o}{.}\PY{n}{xlabel}\PY{p}{(}\PY{l+s+s1}{\PYZsq{}}\PY{l+s+s1}{tempo (s)}\PY{l+s+s1}{\PYZsq{}}\PY{p}{)}
\PY{n}{plt}\PY{o}{.}\PY{n}{ylabel}\PY{p}{(}\PY{l+s+s1}{\PYZsq{}}\PY{l+s+s1}{amplitude}\PY{l+s+s1}{\PYZsq{}}\PY{p}{)}
\PY{n}{plt}\PY{o}{.}\PY{n}{grid}\PY{p}{(}\PY{p}{)}

\PY{n}{plt}\PY{o}{.}\PY{n}{figure}\PY{p}{(}\PY{p}{)}
\PY{n}{plt}\PY{o}{.}\PY{n}{plot}\PY{p}{(}\PY{n}{t}\PY{p}{,} \PY{n}{dente}\PY{p}{,}\PY{l+s+s1}{\PYZsq{}}\PY{l+s+s1}{b}\PY{l+s+s1}{\PYZsq{}}\PY{p}{)}
\PY{n}{plt}\PY{o}{.}\PY{n}{ylim}\PY{p}{(}\PY{n}{dente}\PY{o}{.}\PY{n}{min}\PY{p}{(}\PY{l+m+mi}{0}\PY{p}{)}\PY{o}{\PYZhy{}}\PY{l+m+mf}{0.1}\PY{p}{,} \PY{n}{dente}\PY{o}{.}\PY{n}{max}\PY{p}{(}\PY{l+m+mi}{0}\PY{p}{)}\PY{o}{+}\PY{l+m+mf}{0.1}\PY{p}{)}
\PY{n}{plt}\PY{o}{.}\PY{n}{xlim}\PY{p}{(}\PY{l+m+mi}{0}\PY{p}{,}\PY{n}{t}\PY{o}{.}\PY{n}{max}\PY{p}{(}\PY{l+m+mi}{0}\PY{p}{)}\PY{p}{)}
\PY{n}{plt}\PY{o}{.}\PY{n}{xlabel}\PY{p}{(}\PY{l+s+s1}{\PYZsq{}}\PY{l+s+s1}{tempo (s)}\PY{l+s+s1}{\PYZsq{}}\PY{p}{)}
\PY{n}{plt}\PY{o}{.}\PY{n}{ylabel}\PY{p}{(}\PY{l+s+s1}{\PYZsq{}}\PY{l+s+s1}{amplitude}\PY{l+s+s1}{\PYZsq{}}\PY{p}{)}
\PY{n}{plt}\PY{o}{.}\PY{n}{grid}\PY{p}{(}\PY{p}{)}
\end{Verbatim}
\end{tcolorbox}

    \begin{center}
    \adjustimage{max size={0.9\linewidth}{0.9\paperheight}}{output_3_0.png}
    \end{center}
    { \hspace*{\fill} \\}
    
    \begin{center}
    \adjustimage{max size={0.9\linewidth}{0.9\paperheight}}{output_3_1.png}
    \end{center}
    { \hspace*{\fill} \\}
    
    \begin{center}
    \adjustimage{max size={0.9\linewidth}{0.9\paperheight}}{output_3_2.png}
    \end{center}
    { \hspace*{\fill} \\}
    
    \hypertarget{cuxe1lculo-dos-coeficientes-de-fourier-via-integrauxe7uxe3o-numuxe9rica}{%
\subsubsection{Cálculo dos coeficientes de Fourier via integração
numérica}\label{cuxe1lculo-dos-coeficientes-de-fourier-via-integrauxe7uxe3o-numuxe9rica}}

Primeiramente, vamos definir uma função para calcular numericamente os
coeficientes \(a_n\) e \(b_n\) da série de Fourier para uma função
periódica \(f(t)\) qualquer.

    \begin{tcolorbox}[breakable, size=fbox, boxrule=1pt, pad at break*=1mm,colback=cellbackground, colframe=cellborder]
\prompt{In}{incolor}{170}{\boxspacing}
\begin{Verbatim}[commandchars=\\\{\}]
\PY{k}{def} \PY{n+nf}{fourierCoeff}\PY{p}{(}\PY{n}{t}\PY{p}{,}\PY{n}{f}\PY{p}{,}\PY{n}{P}\PY{p}{,}\PY{n}{n}\PY{p}{)}\PY{p}{:}
    \PY{l+s+sd}{\PYZdq{}\PYZdq{}\PYZdq{}}
\PY{l+s+sd}{     t  : vetor de instantes de tempo contendo pelo menos um período completo da função [segundos]}
\PY{l+s+sd}{     f  : vetor de valores de f(t) calculados para cada instante em t }
\PY{l+s+sd}{     P  : período fundamental de f [segundos]}
\PY{l+s+sd}{     n  : ordem do coeficiente de Fourier desejado [número inteiro]}
\PY{l+s+sd}{     an : coeficiente an}
\PY{l+s+sd}{     bn : coeficiente bn}
\PY{l+s+sd}{     }
\PY{l+s+sd}{    \PYZdq{}\PYZdq{}\PYZdq{}}
    \PY{n}{dt} \PY{o}{=} \PY{n}{t}\PY{p}{[}\PY{l+m+mi}{1}\PY{p}{]}\PY{o}{\PYZhy{}}\PY{n}{t}\PY{p}{[}\PY{l+m+mi}{0}\PY{p}{]}     \PY{c+c1}{\PYZsh{} período de amostragem [passo de integração]}
    \PY{n}{N}  \PY{o}{=} \PY{n}{np}\PY{o}{.}\PY{n}{ceil}\PY{p}{(}\PY{n}{P}\PY{o}{/}\PY{n}{dt}\PY{p}{)} \PY{c+c1}{\PYZsh{} número de amostras correspondente a um período completo da onda   }
    \PY{n}{an} \PY{o}{=} \PY{l+m+mi}{0}
    \PY{n}{bn} \PY{o}{=} \PY{l+m+mi}{0}  
   
    \PY{c+c1}{\PYZsh{} integração numérica dos coeficientes an e bn}
    \PY{k}{for} \PY{n}{ind} \PY{o+ow}{in} \PY{n+nb}{range}\PY{p}{(}\PY{l+m+mi}{0}\PY{p}{,} \PY{n+nb}{int}\PY{p}{(}\PY{n}{N}\PY{p}{)}\PY{p}{)}\PY{p}{:}
        \PY{n}{an} \PY{o}{=} \PY{n}{an} \PY{o}{+} \PY{n}{f}\PY{p}{[}\PY{n}{ind}\PY{p}{]}\PY{o}{*}\PY{n}{cos}\PY{p}{(}\PY{l+m+mi}{2}\PY{o}{*}\PY{n}{np}\PY{o}{.}\PY{n}{pi}\PY{o}{*}\PY{n}{t}\PY{p}{[}\PY{n}{ind}\PY{p}{]}\PY{o}{*}\PY{n}{n}\PY{o}{/}\PY{n}{P}\PY{p}{)}\PY{o}{*}\PY{n}{dt}
        \PY{n}{bn} \PY{o}{=} \PY{n}{bn} \PY{o}{+} \PY{n}{f}\PY{p}{[}\PY{n}{ind}\PY{p}{]}\PY{o}{*}\PY{n}{sin}\PY{p}{(}\PY{l+m+mi}{2}\PY{o}{*}\PY{n}{np}\PY{o}{.}\PY{n}{pi}\PY{o}{*}\PY{n}{t}\PY{p}{[}\PY{n}{ind}\PY{p}{]}\PY{o}{*}\PY{n}{n}\PY{o}{/}\PY{n}{P}\PY{p}{)}\PY{o}{*}\PY{n}{dt}
    
    \PY{n}{an} \PY{o}{=} \PY{n}{an}\PY{o}{*}\PY{l+m+mi}{2}\PY{o}{/}\PY{n}{P}
    \PY{n}{bn} \PY{o}{=} \PY{n}{bn}\PY{o}{*}\PY{l+m+mi}{2}\PY{o}{/}\PY{n}{P}
    
    \PY{k}{return} \PY{n}{an}\PY{p}{,} \PY{n}{bn}
\end{Verbatim}
\end{tcolorbox}

    \hypertarget{aproximauxe7uxe3o-via-suxe9rie-de-fourier-para-a-onda-quadrada}{%
\subsubsection{Aproximação via série de Fourier para a onda
quadrada}\label{aproximauxe7uxe3o-via-suxe9rie-de-fourier-para-a-onda-quadrada}}

    \begin{tcolorbox}[breakable, size=fbox, boxrule=1pt, pad at break*=1mm,colback=cellbackground, colframe=cellborder]
\prompt{In}{incolor}{171}{\boxspacing}
\begin{Verbatim}[commandchars=\\\{\}]
\PY{c+c1}{\PYZsh{} Coeficientes de Fourier da onda quadrada}

\PY{n}{ncoeffs} \PY{o}{=} \PY{l+m+mi}{20}  \PY{c+c1}{\PYZsh{} número de componentes harmônicos (incluindo componente dc, n=0)}

\PY{n}{quad\PYZus{}an} \PY{o}{=} \PY{n}{np}\PY{o}{.}\PY{n}{zeros}\PY{p}{(}\PY{p}{(}\PY{n}{ncoeffs}\PY{p}{,} \PY{l+m+mi}{1}\PY{p}{)}\PY{p}{)}\PY{p}{;}
\PY{n}{quad\PYZus{}bn} \PY{o}{=} \PY{n}{np}\PY{o}{.}\PY{n}{zeros}\PY{p}{(}\PY{p}{(}\PY{n}{ncoeffs}\PY{p}{,} \PY{l+m+mi}{1}\PY{p}{)}\PY{p}{)}\PY{p}{;}
\PY{n}{xf}      \PY{o}{=} \PY{n}{np}\PY{o}{.}\PY{n}{zeros}\PY{p}{(}\PY{p}{(}\PY{n}{ncoeffs}\PY{p}{,} \PY{l+m+mi}{1}\PY{p}{)}\PY{p}{)}\PY{p}{;}

\PY{k}{for} \PY{n}{n} \PY{o+ow}{in} \PY{n+nb}{range}\PY{p}{(}\PY{l+m+mi}{0}\PY{p}{,}\PY{n}{ncoeffs}\PY{p}{)}\PY{p}{:} \PY{c+c1}{\PYZsh{} calcula coeficientes de Fourier para n=0 até n=ncoeffs\PYZhy{}1}
    \PY{n}{quad\PYZus{}an}\PY{p}{[}\PY{n}{n}\PY{p}{]}\PY{p}{,} \PY{n}{quad\PYZus{}bn}\PY{p}{[}\PY{n}{n}\PY{p}{]} \PY{o}{=} \PY{n}{fourierCoeff}\PY{p}{(}\PY{n}{t}\PY{p}{,} \PY{n}{quad} \PY{p}{,} \PY{l+m+mi}{1}\PY{o}{/}\PY{n}{f0}\PY{p}{,} \PY{n}{n}\PY{p}{)}
    \PY{n}{xf}\PY{p}{[}\PY{n}{n}\PY{p}{]} \PY{o}{=} \PY{n}{n}\PY{o}{*}\PY{n}{f0}
    
\PY{n}{plt}\PY{o}{.}\PY{n}{figure}\PY{p}{(}\PY{p}{)}
\PY{p}{(}\PY{n}{markers}\PY{p}{,} \PY{n}{stemlines}\PY{p}{,} \PY{n}{baseline}\PY{p}{)} \PY{o}{=} \PY{n}{plt}\PY{o}{.}\PY{n}{stem}\PY{p}{(}\PY{n}{xf}\PY{p}{,} \PY{n}{quad\PYZus{}an}\PY{p}{,}\PY{l+s+s1}{\PYZsq{}}\PY{l+s+s1}{b}\PY{l+s+s1}{\PYZsq{}}\PY{p}{,} \PY{n}{use\PYZus{}line\PYZus{}collection}\PY{o}{=}\PY{k+kc}{True}\PY{p}{,}\PY{n}{label} \PY{o}{=} \PY{l+s+s1}{\PYZsq{}}\PY{l+s+s1}{coeficientes an}\PY{l+s+s1}{\PYZsq{}}\PY{p}{)}
\PY{n}{plt}\PY{o}{.}\PY{n}{setp}\PY{p}{(}\PY{n}{baseline}\PY{p}{,} \PY{n}{visible}\PY{o}{=}\PY{k+kc}{False}\PY{p}{)}
\PY{n}{plt}\PY{o}{.}\PY{n}{setp}\PY{p}{(}\PY{n}{markers}\PY{p}{,} \PY{n}{marker}\PY{o}{=}\PY{l+s+s1}{\PYZsq{}}\PY{l+s+s1}{o}\PY{l+s+s1}{\PYZsq{}}\PY{p}{,} \PY{n}{markersize}\PY{o}{=}\PY{l+m+mi}{5}\PY{p}{,} \PY{n}{markeredgecolor}\PY{o}{=}\PY{l+s+s2}{\PYZdq{}}\PY{l+s+s2}{b}\PY{l+s+s2}{\PYZdq{}}\PY{p}{,} \PY{n}{markeredgewidth}\PY{o}{=}\PY{l+m+mi}{2}\PY{p}{)}
\PY{n}{plt}\PY{o}{.}\PY{n}{setp}\PY{p}{(}\PY{n}{stemlines}\PY{p}{,} \PY{n}{linestyle}\PY{o}{=}\PY{l+s+s2}{\PYZdq{}}\PY{l+s+s2}{\PYZhy{}\PYZhy{}}\PY{l+s+s2}{\PYZdq{}}\PY{p}{,} \PY{n}{color}\PY{o}{=}\PY{l+s+s2}{\PYZdq{}}\PY{l+s+s2}{b}\PY{l+s+s2}{\PYZdq{}}\PY{p}{,} \PY{n}{linewidth}\PY{o}{=}\PY{l+m+mf}{0.5} \PY{p}{)}

\PY{p}{(}\PY{n}{markers}\PY{p}{,} \PY{n}{stemlines}\PY{p}{,} \PY{n}{baseline}\PY{p}{)} \PY{o}{=} \PY{n}{plt}\PY{o}{.}\PY{n}{stem}\PY{p}{(}\PY{n}{xf}\PY{p}{,} \PY{n}{quad\PYZus{}bn}\PY{p}{,}\PY{l+s+s1}{\PYZsq{}}\PY{l+s+s1}{r}\PY{l+s+s1}{\PYZsq{}}\PY{p}{,} \PY{n}{use\PYZus{}line\PYZus{}collection}\PY{o}{=}\PY{k+kc}{True}\PY{p}{,}\PY{n}{label} \PY{o}{=} \PY{l+s+s1}{\PYZsq{}}\PY{l+s+s1}{coeficientes bn}\PY{l+s+s1}{\PYZsq{}}\PY{p}{)}
\PY{n}{plt}\PY{o}{.}\PY{n}{setp}\PY{p}{(}\PY{n}{baseline}\PY{p}{,} \PY{n}{visible}\PY{o}{=}\PY{k+kc}{False}\PY{p}{)}
\PY{n}{plt}\PY{o}{.}\PY{n}{setp}\PY{p}{(}\PY{n}{markers}\PY{p}{,} \PY{n}{marker}\PY{o}{=}\PY{l+s+s1}{\PYZsq{}}\PY{l+s+s1}{x}\PY{l+s+s1}{\PYZsq{}}\PY{p}{,} \PY{n}{markersize}\PY{o}{=}\PY{l+m+mi}{5}\PY{p}{,} \PY{n}{markeredgecolor}\PY{o}{=}\PY{l+s+s2}{\PYZdq{}}\PY{l+s+s2}{r}\PY{l+s+s2}{\PYZdq{}}\PY{p}{,} \PY{n}{markeredgewidth}\PY{o}{=}\PY{l+m+mi}{2}\PY{p}{)}
\PY{n}{plt}\PY{o}{.}\PY{n}{setp}\PY{p}{(}\PY{n}{stemlines}\PY{p}{,} \PY{n}{linestyle}\PY{o}{=}\PY{l+s+s2}{\PYZdq{}}\PY{l+s+s2}{\PYZhy{}\PYZhy{}}\PY{l+s+s2}{\PYZdq{}}\PY{p}{,} \PY{n}{color}\PY{o}{=}\PY{l+s+s2}{\PYZdq{}}\PY{l+s+s2}{r}\PY{l+s+s2}{\PYZdq{}}\PY{p}{,} \PY{n}{linewidth}\PY{o}{=}\PY{l+m+mf}{0.5} \PY{p}{)}

\PY{n}{plt}\PY{o}{.}\PY{n}{title}\PY{p}{(}\PY{l+s+s1}{\PYZsq{}}\PY{l+s+s1}{Onda quadrada}\PY{l+s+s1}{\PYZsq{}}\PY{p}{)}
\PY{n}{plt}\PY{o}{.}\PY{n}{legend}\PY{p}{(}\PY{p}{)}
\PY{n}{plt}\PY{o}{.}\PY{n}{xlim}\PY{p}{(}\PY{l+m+mi}{0}\PY{p}{,}\PY{n}{xf}\PY{o}{.}\PY{n}{max}\PY{p}{(}\PY{l+m+mi}{0}\PY{p}{)}\PY{p}{)}
\PY{n}{plt}\PY{o}{.}\PY{n}{ylim}\PY{p}{(}\PY{o}{\PYZhy{}}\PY{l+m+mi}{2}\PY{p}{,}\PY{l+m+mi}{2}\PY{p}{)}
\PY{n}{plt}\PY{o}{.}\PY{n}{xlabel}\PY{p}{(}\PY{l+s+s1}{\PYZsq{}}\PY{l+s+s1}{freq (Hz)}\PY{l+s+s1}{\PYZsq{}}\PY{p}{)}
\PY{n}{plt}\PY{o}{.}\PY{n}{ylabel}\PY{p}{(}\PY{l+s+s1}{\PYZsq{}}\PY{l+s+s1}{amplitude}\PY{l+s+s1}{\PYZsq{}}\PY{p}{)}
\PY{n}{plt}\PY{o}{.}\PY{n}{grid}\PY{p}{(}\PY{p}{)}
\end{Verbatim}
\end{tcolorbox}

    \begin{center}
    \adjustimage{max size={0.9\linewidth}{0.9\paperheight}}{output_7_0.png}
    \end{center}
    { \hspace*{\fill} \\}
    
    \begin{tcolorbox}[breakable, size=fbox, boxrule=1pt, pad at break*=1mm,colback=cellbackground, colframe=cellborder]
\prompt{In}{incolor}{172}{\boxspacing}
\begin{Verbatim}[commandchars=\\\{\}]
\PY{c+c1}{\PYZsh{} Aproximação da onda quadrada via somatório de harmônicas da série de Fourier}

\PY{n}{ncoeff} \PY{o}{=} \PY{l+m+mi}{20} \PY{c+c1}{\PYZsh{} número de componentes harmônicos (incluindo componente dc, n=0)}

\PY{n}{quad\PYZus{}aprox} \PY{o}{=} \PY{l+m+mi}{0}
\PY{k}{for} \PY{n}{n} \PY{o+ow}{in} \PY{n+nb}{range}\PY{p}{(}\PY{l+m+mi}{0}\PY{p}{,}\PY{n}{ncoeff}\PY{p}{)}\PY{p}{:}
    \PY{k}{if} \PY{n}{n} \PY{o}{!=} \PY{l+m+mi}{0}\PY{p}{:}
        \PY{n}{quad\PYZus{}aprox} \PY{o}{=} \PY{n}{quad\PYZus{}aprox} \PY{o}{+} \PY{n}{quad\PYZus{}an}\PY{p}{[}\PY{n}{xf}\PY{o}{==}\PY{n}{n}\PY{o}{*}\PY{n}{f0}\PY{p}{]}\PY{o}{*}\PY{n}{cos}\PY{p}{(}\PY{l+m+mi}{2}\PY{o}{*}\PY{n}{π}\PY{o}{*}\PY{n}{n}\PY{o}{*}\PY{n}{f0}\PY{o}{*}\PY{n}{t}\PY{p}{)}\PYZbs{}
                                \PY{o}{+} \PY{n}{quad\PYZus{}bn}\PY{p}{[}\PY{n}{xf}\PY{o}{==}\PY{n}{n}\PY{o}{*}\PY{n}{f0}\PY{p}{]}\PY{o}{*}\PY{n}{sin}\PY{p}{(}\PY{l+m+mi}{2}\PY{o}{*}\PY{n}{π}\PY{o}{*}\PY{n}{n}\PY{o}{*}\PY{n}{f0}\PY{o}{*}\PY{n}{t}\PY{p}{)} \PY{c+c1}{\PYZsh{} soma n\PYZhy{}ésimo componente harmônico}
    \PY{k}{else}\PY{p}{:}
        \PY{n}{quad\PYZus{}aprox} \PY{o}{=} \PY{n}{quad\PYZus{}an}\PY{p}{[}\PY{n}{xf}\PY{o}{==}\PY{n}{n}\PY{o}{*}\PY{n}{f0}\PY{p}{]}\PY{o}{/}\PY{l+m+mi}{2}

\PY{n}{plt}\PY{o}{.}\PY{n}{plot}\PY{p}{(}\PY{n}{t}\PY{p}{,} \PY{n}{quad}\PY{p}{,}\PY{l+s+s1}{\PYZsq{}}\PY{l+s+s1}{k\PYZhy{}\PYZhy{}}\PY{l+s+s1}{\PYZsq{}}\PY{p}{,}\PY{n}{label} \PY{o}{=} \PY{l+s+s1}{\PYZsq{}}\PY{l+s+s1}{original}\PY{l+s+s1}{\PYZsq{}}\PY{p}{)}        
\PY{n}{plt}\PY{o}{.}\PY{n}{plot}\PY{p}{(}\PY{n}{t}\PY{p}{,}\PY{n}{quad\PYZus{}aprox}\PY{p}{,}\PY{l+s+s1}{\PYZsq{}}\PY{l+s+s1}{b}\PY{l+s+s1}{\PYZsq{}}\PY{p}{,}\PY{n}{label} \PY{o}{=} \PY{l+s+s1}{\PYZsq{}}\PY{l+s+s1}{aprox}\PY{l+s+s1}{\PYZsq{}}\PY{p}{)}
\PY{n}{plt}\PY{o}{.}\PY{n}{xlim}\PY{p}{(}\PY{l+m+mi}{0}\PY{p}{,}\PY{n}{t}\PY{o}{.}\PY{n}{max}\PY{p}{(}\PY{l+m+mi}{0}\PY{p}{)}\PY{p}{)}
\PY{n}{plt}\PY{o}{.}\PY{n}{xlabel}\PY{p}{(}\PY{l+s+s1}{\PYZsq{}}\PY{l+s+s1}{tempo (s)}\PY{l+s+s1}{\PYZsq{}}\PY{p}{)}
\PY{n}{plt}\PY{o}{.}\PY{n}{ylabel}\PY{p}{(}\PY{l+s+s1}{\PYZsq{}}\PY{l+s+s1}{amplitude}\PY{l+s+s1}{\PYZsq{}}\PY{p}{)}
\PY{n}{plt}\PY{o}{.}\PY{n}{legend}\PY{p}{(}\PY{p}{)}
\PY{n}{plt}\PY{o}{.}\PY{n}{grid}\PY{p}{(}\PY{p}{)}
\end{Verbatim}
\end{tcolorbox}

    \begin{center}
    \adjustimage{max size={0.9\linewidth}{0.9\paperheight}}{output_8_0.png}
    \end{center}
    { \hspace*{\fill} \\}
    
    \hypertarget{aproximauxe7uxe3o-via-suxe9rie-de-fourier-para-a-onda-triangular}{%
\subsubsection{Aproximação via série de Fourier para a onda
triangular}\label{aproximauxe7uxe3o-via-suxe9rie-de-fourier-para-a-onda-triangular}}

    \begin{tcolorbox}[breakable, size=fbox, boxrule=1pt, pad at break*=1mm,colback=cellbackground, colframe=cellborder]
\prompt{In}{incolor}{173}{\boxspacing}
\begin{Verbatim}[commandchars=\\\{\}]
\PY{c+c1}{\PYZsh{} Coeficientes de Fourier da onda triangular}

\PY{n}{ncoeffs} \PY{o}{=} \PY{l+m+mi}{10} \PY{c+c1}{\PYZsh{} número de componentes harmônicos (incluindo componente dc, n=0)}

\PY{n}{triang\PYZus{}an} \PY{o}{=} \PY{n}{np}\PY{o}{.}\PY{n}{zeros}\PY{p}{(}\PY{p}{(}\PY{n}{ncoeffs}\PY{p}{,} \PY{l+m+mi}{1}\PY{p}{)}\PY{p}{)}\PY{p}{;}
\PY{n}{triang\PYZus{}bn} \PY{o}{=} \PY{n}{np}\PY{o}{.}\PY{n}{zeros}\PY{p}{(}\PY{p}{(}\PY{n}{ncoeffs}\PY{p}{,} \PY{l+m+mi}{1}\PY{p}{)}\PY{p}{)}\PY{p}{;}
\PY{n}{xf}        \PY{o}{=} \PY{n}{np}\PY{o}{.}\PY{n}{zeros}\PY{p}{(}\PY{p}{(}\PY{n}{ncoeffs}\PY{p}{,} \PY{l+m+mi}{1}\PY{p}{)}\PY{p}{)}\PY{p}{;}

\PY{k}{for} \PY{n}{n} \PY{o+ow}{in} \PY{n+nb}{range}\PY{p}{(}\PY{l+m+mi}{0}\PY{p}{,}\PY{n}{ncoeffs}\PY{p}{)}\PY{p}{:} \PY{c+c1}{\PYZsh{} calcula coeficientes de Fourier para n=0 até n=ncoeffs\PYZhy{}1}
    \PY{n}{triang\PYZus{}an}\PY{p}{[}\PY{n}{n}\PY{p}{]}\PY{p}{,} \PY{n}{triang\PYZus{}bn}\PY{p}{[}\PY{n}{n}\PY{p}{]} \PY{o}{=} \PY{n}{fourierCoeff}\PY{p}{(}\PY{n}{t}\PY{p}{,} \PY{n}{triang} \PY{p}{,} \PY{l+m+mi}{1}\PY{o}{/}\PY{n}{f0}\PY{p}{,} \PY{n}{n}\PY{p}{)}
    \PY{n}{xf}\PY{p}{[}\PY{n}{n}\PY{p}{]} \PY{o}{=} \PY{n}{n}\PY{o}{*}\PY{n}{f0}
    
\PY{n}{plt}\PY{o}{.}\PY{n}{figure}\PY{p}{(}\PY{p}{)}
\PY{p}{(}\PY{n}{markers}\PY{p}{,} \PY{n}{stemlines}\PY{p}{,} \PY{n}{baseline}\PY{p}{)} \PY{o}{=} \PY{n}{plt}\PY{o}{.}\PY{n}{stem}\PY{p}{(}\PY{n}{xf}\PY{p}{,} \PY{n}{triang\PYZus{}an}\PY{p}{,}\PY{l+s+s1}{\PYZsq{}}\PY{l+s+s1}{b}\PY{l+s+s1}{\PYZsq{}}\PY{p}{,} \PY{n}{use\PYZus{}line\PYZus{}collection}\PY{o}{=}\PY{k+kc}{True}\PY{p}{,}\PY{n}{label} \PY{o}{=} \PY{l+s+s1}{\PYZsq{}}\PY{l+s+s1}{coeficientes an}\PY{l+s+s1}{\PYZsq{}}\PY{p}{)}
\PY{n}{plt}\PY{o}{.}\PY{n}{setp}\PY{p}{(}\PY{n}{baseline}\PY{p}{,} \PY{n}{visible}\PY{o}{=}\PY{k+kc}{False}\PY{p}{)}
\PY{n}{plt}\PY{o}{.}\PY{n}{setp}\PY{p}{(}\PY{n}{markers}\PY{p}{,} \PY{n}{marker}\PY{o}{=}\PY{l+s+s1}{\PYZsq{}}\PY{l+s+s1}{o}\PY{l+s+s1}{\PYZsq{}}\PY{p}{,} \PY{n}{markersize}\PY{o}{=}\PY{l+m+mi}{5}\PY{p}{,} \PY{n}{markeredgecolor}\PY{o}{=}\PY{l+s+s2}{\PYZdq{}}\PY{l+s+s2}{b}\PY{l+s+s2}{\PYZdq{}}\PY{p}{,} \PY{n}{markeredgewidth}\PY{o}{=}\PY{l+m+mi}{2}\PY{p}{)}
\PY{n}{plt}\PY{o}{.}\PY{n}{setp}\PY{p}{(}\PY{n}{stemlines}\PY{p}{,} \PY{n}{linestyle}\PY{o}{=}\PY{l+s+s2}{\PYZdq{}}\PY{l+s+s2}{\PYZhy{}\PYZhy{}}\PY{l+s+s2}{\PYZdq{}}\PY{p}{,} \PY{n}{color}\PY{o}{=}\PY{l+s+s2}{\PYZdq{}}\PY{l+s+s2}{b}\PY{l+s+s2}{\PYZdq{}}\PY{p}{,} \PY{n}{linewidth}\PY{o}{=}\PY{l+m+mf}{0.5} \PY{p}{)}

\PY{p}{(}\PY{n}{markers}\PY{p}{,} \PY{n}{stemlines}\PY{p}{,} \PY{n}{baseline}\PY{p}{)} \PY{o}{=} \PY{n}{plt}\PY{o}{.}\PY{n}{stem}\PY{p}{(}\PY{n}{xf}\PY{p}{,} \PY{n}{triang\PYZus{}bn}\PY{p}{,}\PY{l+s+s1}{\PYZsq{}}\PY{l+s+s1}{r}\PY{l+s+s1}{\PYZsq{}}\PY{p}{,} \PY{n}{use\PYZus{}line\PYZus{}collection}\PY{o}{=}\PY{k+kc}{True}\PY{p}{,}\PY{n}{label} \PY{o}{=} \PY{l+s+s1}{\PYZsq{}}\PY{l+s+s1}{coeficientes bn}\PY{l+s+s1}{\PYZsq{}}\PY{p}{)}
\PY{n}{plt}\PY{o}{.}\PY{n}{setp}\PY{p}{(}\PY{n}{baseline}\PY{p}{,} \PY{n}{visible}\PY{o}{=}\PY{k+kc}{False}\PY{p}{)}
\PY{n}{plt}\PY{o}{.}\PY{n}{setp}\PY{p}{(}\PY{n}{markers}\PY{p}{,} \PY{n}{marker}\PY{o}{=}\PY{l+s+s1}{\PYZsq{}}\PY{l+s+s1}{x}\PY{l+s+s1}{\PYZsq{}}\PY{p}{,} \PY{n}{markersize}\PY{o}{=}\PY{l+m+mi}{5}\PY{p}{,} \PY{n}{markeredgecolor}\PY{o}{=}\PY{l+s+s2}{\PYZdq{}}\PY{l+s+s2}{r}\PY{l+s+s2}{\PYZdq{}}\PY{p}{,} \PY{n}{markeredgewidth}\PY{o}{=}\PY{l+m+mi}{2}\PY{p}{)}
\PY{n}{plt}\PY{o}{.}\PY{n}{setp}\PY{p}{(}\PY{n}{stemlines}\PY{p}{,} \PY{n}{linestyle}\PY{o}{=}\PY{l+s+s2}{\PYZdq{}}\PY{l+s+s2}{\PYZhy{}\PYZhy{}}\PY{l+s+s2}{\PYZdq{}}\PY{p}{,} \PY{n}{color}\PY{o}{=}\PY{l+s+s2}{\PYZdq{}}\PY{l+s+s2}{r}\PY{l+s+s2}{\PYZdq{}}\PY{p}{,} \PY{n}{linewidth}\PY{o}{=}\PY{l+m+mf}{0.5} \PY{p}{)}

\PY{n}{plt}\PY{o}{.}\PY{n}{title}\PY{p}{(}\PY{l+s+s1}{\PYZsq{}}\PY{l+s+s1}{Onda triangular}\PY{l+s+s1}{\PYZsq{}}\PY{p}{)}
\PY{n}{plt}\PY{o}{.}\PY{n}{legend}\PY{p}{(}\PY{p}{)}
\PY{n}{plt}\PY{o}{.}\PY{n}{xlim}\PY{p}{(}\PY{l+m+mi}{0}\PY{p}{,}\PY{n}{xf}\PY{o}{.}\PY{n}{max}\PY{p}{(}\PY{l+m+mi}{0}\PY{p}{)}\PY{p}{)}
\PY{n}{plt}\PY{o}{.}\PY{n}{ylim}\PY{p}{(}\PY{o}{\PYZhy{}}\PY{l+m+mi}{2}\PY{p}{,}\PY{l+m+mi}{2}\PY{p}{)}
\PY{n}{plt}\PY{o}{.}\PY{n}{xlabel}\PY{p}{(}\PY{l+s+s1}{\PYZsq{}}\PY{l+s+s1}{freq (Hz)}\PY{l+s+s1}{\PYZsq{}}\PY{p}{)}
\PY{n}{plt}\PY{o}{.}\PY{n}{ylabel}\PY{p}{(}\PY{l+s+s1}{\PYZsq{}}\PY{l+s+s1}{amplitude}\PY{l+s+s1}{\PYZsq{}}\PY{p}{)}
\PY{n}{plt}\PY{o}{.}\PY{n}{grid}\PY{p}{(}\PY{p}{)}
\end{Verbatim}
\end{tcolorbox}

    \begin{center}
    \adjustimage{max size={0.9\linewidth}{0.9\paperheight}}{output_10_0.png}
    \end{center}
    { \hspace*{\fill} \\}
    
    \begin{tcolorbox}[breakable, size=fbox, boxrule=1pt, pad at break*=1mm,colback=cellbackground, colframe=cellborder]
\prompt{In}{incolor}{174}{\boxspacing}
\begin{Verbatim}[commandchars=\\\{\}]
\PY{c+c1}{\PYZsh{} Aproximação da onda triangular via somatório de harmônicas da série de Fourier}

\PY{n}{ncoeff} \PY{o}{=} \PY{l+m+mi}{10} \PY{c+c1}{\PYZsh{} número de componentes harmônicos (incluindo componente dc, n=0)}

\PY{n}{triang\PYZus{}aprox} \PY{o}{=} \PY{l+m+mi}{0}
\PY{k}{for} \PY{n}{n} \PY{o+ow}{in} \PY{n+nb}{range}\PY{p}{(}\PY{l+m+mi}{0}\PY{p}{,}\PY{n}{ncoeff}\PY{p}{)}\PY{p}{:}
    \PY{k}{if} \PY{n}{n} \PY{o}{!=} \PY{l+m+mi}{0}\PY{p}{:}
        \PY{n}{triang\PYZus{}aprox} \PY{o}{=} \PY{n}{triang\PYZus{}aprox} \PY{o}{+} \PY{n}{triang\PYZus{}an}\PY{p}{[}\PY{n}{xf}\PY{o}{==}\PY{n}{n}\PY{o}{*}\PY{n}{f0}\PY{p}{]}\PY{o}{*}\PY{n}{cos}\PY{p}{(}\PY{l+m+mi}{2}\PY{o}{*}\PY{n}{π}\PY{o}{*}\PY{n}{n}\PY{o}{*}\PY{n}{f0}\PY{o}{*}\PY{n}{t}\PY{p}{)}\PYZbs{}
                                    \PY{o}{+} \PY{n}{triang\PYZus{}bn}\PY{p}{[}\PY{n}{xf}\PY{o}{==}\PY{n}{n}\PY{o}{*}\PY{n}{f0}\PY{p}{]}\PY{o}{*}\PY{n}{sin}\PY{p}{(}\PY{l+m+mi}{2}\PY{o}{*}\PY{n}{π}\PY{o}{*}\PY{n}{n}\PY{o}{*}\PY{n}{f0}\PY{o}{*}\PY{n}{t}\PY{p}{)} \PY{c+c1}{\PYZsh{} soma n\PYZhy{}ésimo componente harmônico}
    \PY{k}{else}\PY{p}{:}
        \PY{n}{triang\PYZus{}aprox} \PY{o}{=} \PY{n}{triang\PYZus{}an}\PY{p}{[}\PY{n}{xf}\PY{o}{==}\PY{n}{n}\PY{o}{*}\PY{n}{f0}\PY{p}{]}\PY{o}{/}\PY{l+m+mi}{2}

\PY{n}{plt}\PY{o}{.}\PY{n}{plot}\PY{p}{(}\PY{n}{t}\PY{p}{,} \PY{n}{triang}\PY{p}{,}\PY{l+s+s1}{\PYZsq{}}\PY{l+s+s1}{k\PYZhy{}\PYZhy{}}\PY{l+s+s1}{\PYZsq{}}\PY{p}{,}\PY{n}{label} \PY{o}{=} \PY{l+s+s1}{\PYZsq{}}\PY{l+s+s1}{original}\PY{l+s+s1}{\PYZsq{}}\PY{p}{)}        
\PY{n}{plt}\PY{o}{.}\PY{n}{plot}\PY{p}{(}\PY{n}{t}\PY{p}{,}\PY{n}{triang\PYZus{}aprox}\PY{p}{,}\PY{l+s+s1}{\PYZsq{}}\PY{l+s+s1}{b}\PY{l+s+s1}{\PYZsq{}}\PY{p}{,}\PY{n}{label} \PY{o}{=} \PY{l+s+s1}{\PYZsq{}}\PY{l+s+s1}{aprox}\PY{l+s+s1}{\PYZsq{}}\PY{p}{)}
\PY{n}{plt}\PY{o}{.}\PY{n}{xlim}\PY{p}{(}\PY{l+m+mi}{0}\PY{p}{,}\PY{n}{t}\PY{o}{.}\PY{n}{max}\PY{p}{(}\PY{l+m+mi}{0}\PY{p}{)}\PY{p}{)}
\PY{n}{plt}\PY{o}{.}\PY{n}{xlabel}\PY{p}{(}\PY{l+s+s1}{\PYZsq{}}\PY{l+s+s1}{tempo (s)}\PY{l+s+s1}{\PYZsq{}}\PY{p}{)}
\PY{n}{plt}\PY{o}{.}\PY{n}{ylabel}\PY{p}{(}\PY{l+s+s1}{\PYZsq{}}\PY{l+s+s1}{amplitude}\PY{l+s+s1}{\PYZsq{}}\PY{p}{)}
\PY{n}{plt}\PY{o}{.}\PY{n}{legend}\PY{p}{(}\PY{p}{)}
\PY{n}{plt}\PY{o}{.}\PY{n}{grid}\PY{p}{(}\PY{p}{)}
\end{Verbatim}
\end{tcolorbox}

    \begin{center}
    \adjustimage{max size={0.9\linewidth}{0.9\paperheight}}{output_11_0.png}
    \end{center}
    { \hspace*{\fill} \\}
    
    \hypertarget{aproximauxe7uxe3o-via-suxe9rie-de-fourier-para-a-onda-dente-de-serra}{%
\subsubsection{Aproximação via série de Fourier para a onda dente de
serra}\label{aproximauxe7uxe3o-via-suxe9rie-de-fourier-para-a-onda-dente-de-serra}}

    \begin{tcolorbox}[breakable, size=fbox, boxrule=1pt, pad at break*=1mm,colback=cellbackground, colframe=cellborder]
\prompt{In}{incolor}{175}{\boxspacing}
\begin{Verbatim}[commandchars=\\\{\}]
\PY{c+c1}{\PYZsh{} Coeficientes de Fourier da onda dente de serra}

\PY{n}{ncoeffs} \PY{o}{=} \PY{l+m+mi}{10} \PY{c+c1}{\PYZsh{} número de componentes harmônicos (incluindo componente dc, n=0)}

\PY{n}{dente\PYZus{}an} \PY{o}{=} \PY{n}{np}\PY{o}{.}\PY{n}{zeros}\PY{p}{(}\PY{p}{(}\PY{n}{ncoeffs}\PY{p}{,} \PY{l+m+mi}{1}\PY{p}{)}\PY{p}{)}\PY{p}{;}
\PY{n}{dente\PYZus{}bn} \PY{o}{=} \PY{n}{np}\PY{o}{.}\PY{n}{zeros}\PY{p}{(}\PY{p}{(}\PY{n}{ncoeffs}\PY{p}{,} \PY{l+m+mi}{1}\PY{p}{)}\PY{p}{)}\PY{p}{;}
\PY{n}{xf}       \PY{o}{=} \PY{n}{np}\PY{o}{.}\PY{n}{zeros}\PY{p}{(}\PY{p}{(}\PY{n}{ncoeffs}\PY{p}{,} \PY{l+m+mi}{1}\PY{p}{)}\PY{p}{)}\PY{p}{;}

\PY{k}{for} \PY{n}{n} \PY{o+ow}{in} \PY{n+nb}{range}\PY{p}{(}\PY{l+m+mi}{0}\PY{p}{,}\PY{n}{ncoeffs}\PY{p}{)}\PY{p}{:} \PY{c+c1}{\PYZsh{} calcula coeficientes de Fourier para n=0 até n=ncoeffs\PYZhy{}1}
    \PY{n}{dente\PYZus{}an}\PY{p}{[}\PY{n}{n}\PY{p}{]}\PY{p}{,} \PY{n}{dente\PYZus{}bn}\PY{p}{[}\PY{n}{n}\PY{p}{]} \PY{o}{=} \PY{n}{fourierCoeff}\PY{p}{(}\PY{n}{t}\PY{p}{,} \PY{n}{dente}\PY{p}{,} \PY{l+m+mi}{1}\PY{o}{/}\PY{n}{f0}\PY{p}{,} \PY{n}{n}\PY{p}{)}
    \PY{n}{xf}\PY{p}{[}\PY{n}{n}\PY{p}{]} \PY{o}{=} \PY{n}{n}\PY{o}{*}\PY{n}{f0}
    
\PY{n}{plt}\PY{o}{.}\PY{n}{figure}\PY{p}{(}\PY{p}{)}
\PY{p}{(}\PY{n}{markers}\PY{p}{,} \PY{n}{stemlines}\PY{p}{,} \PY{n}{baseline}\PY{p}{)} \PY{o}{=} \PY{n}{plt}\PY{o}{.}\PY{n}{stem}\PY{p}{(}\PY{n}{xf}\PY{p}{,} \PY{n}{dente\PYZus{}an}\PY{p}{,}\PY{l+s+s1}{\PYZsq{}}\PY{l+s+s1}{b}\PY{l+s+s1}{\PYZsq{}}\PY{p}{,} \PY{n}{use\PYZus{}line\PYZus{}collection}\PY{o}{=}\PY{k+kc}{True}\PY{p}{,}\PY{n}{label} \PY{o}{=} \PY{l+s+s1}{\PYZsq{}}\PY{l+s+s1}{coeficientes an}\PY{l+s+s1}{\PYZsq{}}\PY{p}{)}
\PY{n}{plt}\PY{o}{.}\PY{n}{setp}\PY{p}{(}\PY{n}{baseline}\PY{p}{,} \PY{n}{visible}\PY{o}{=}\PY{k+kc}{False}\PY{p}{)}
\PY{n}{plt}\PY{o}{.}\PY{n}{setp}\PY{p}{(}\PY{n}{markers}\PY{p}{,} \PY{n}{marker}\PY{o}{=}\PY{l+s+s1}{\PYZsq{}}\PY{l+s+s1}{o}\PY{l+s+s1}{\PYZsq{}}\PY{p}{,} \PY{n}{markersize}\PY{o}{=}\PY{l+m+mi}{5}\PY{p}{,} \PY{n}{markeredgecolor}\PY{o}{=}\PY{l+s+s2}{\PYZdq{}}\PY{l+s+s2}{b}\PY{l+s+s2}{\PYZdq{}}\PY{p}{,} \PY{n}{markeredgewidth}\PY{o}{=}\PY{l+m+mi}{2}\PY{p}{)}
\PY{n}{plt}\PY{o}{.}\PY{n}{setp}\PY{p}{(}\PY{n}{stemlines}\PY{p}{,} \PY{n}{linestyle}\PY{o}{=}\PY{l+s+s2}{\PYZdq{}}\PY{l+s+s2}{\PYZhy{}\PYZhy{}}\PY{l+s+s2}{\PYZdq{}}\PY{p}{,} \PY{n}{color}\PY{o}{=}\PY{l+s+s2}{\PYZdq{}}\PY{l+s+s2}{b}\PY{l+s+s2}{\PYZdq{}}\PY{p}{,} \PY{n}{linewidth}\PY{o}{=}\PY{l+m+mf}{0.5} \PY{p}{)}

\PY{p}{(}\PY{n}{markers}\PY{p}{,} \PY{n}{stemlines}\PY{p}{,} \PY{n}{baseline}\PY{p}{)} \PY{o}{=} \PY{n}{plt}\PY{o}{.}\PY{n}{stem}\PY{p}{(}\PY{n}{xf}\PY{p}{,} \PY{n}{dente\PYZus{}bn}\PY{p}{,}\PY{l+s+s1}{\PYZsq{}}\PY{l+s+s1}{r}\PY{l+s+s1}{\PYZsq{}}\PY{p}{,} \PY{n}{use\PYZus{}line\PYZus{}collection}\PY{o}{=}\PY{k+kc}{True}\PY{p}{,}\PY{n}{label} \PY{o}{=} \PY{l+s+s1}{\PYZsq{}}\PY{l+s+s1}{coeficientes bn}\PY{l+s+s1}{\PYZsq{}}\PY{p}{)}
\PY{n}{plt}\PY{o}{.}\PY{n}{setp}\PY{p}{(}\PY{n}{baseline}\PY{p}{,} \PY{n}{visible}\PY{o}{=}\PY{k+kc}{False}\PY{p}{)}
\PY{n}{plt}\PY{o}{.}\PY{n}{setp}\PY{p}{(}\PY{n}{markers}\PY{p}{,} \PY{n}{marker}\PY{o}{=}\PY{l+s+s1}{\PYZsq{}}\PY{l+s+s1}{x}\PY{l+s+s1}{\PYZsq{}}\PY{p}{,} \PY{n}{markersize}\PY{o}{=}\PY{l+m+mi}{5}\PY{p}{,} \PY{n}{markeredgecolor}\PY{o}{=}\PY{l+s+s2}{\PYZdq{}}\PY{l+s+s2}{r}\PY{l+s+s2}{\PYZdq{}}\PY{p}{,} \PY{n}{markeredgewidth}\PY{o}{=}\PY{l+m+mi}{2}\PY{p}{)}
\PY{n}{plt}\PY{o}{.}\PY{n}{setp}\PY{p}{(}\PY{n}{stemlines}\PY{p}{,} \PY{n}{linestyle}\PY{o}{=}\PY{l+s+s2}{\PYZdq{}}\PY{l+s+s2}{\PYZhy{}\PYZhy{}}\PY{l+s+s2}{\PYZdq{}}\PY{p}{,} \PY{n}{color}\PY{o}{=}\PY{l+s+s2}{\PYZdq{}}\PY{l+s+s2}{r}\PY{l+s+s2}{\PYZdq{}}\PY{p}{,} \PY{n}{linewidth}\PY{o}{=}\PY{l+m+mf}{0.5} \PY{p}{)}

\PY{n}{plt}\PY{o}{.}\PY{n}{title}\PY{p}{(}\PY{l+s+s1}{\PYZsq{}}\PY{l+s+s1}{Onda dente de serra}\PY{l+s+s1}{\PYZsq{}}\PY{p}{)}
\PY{n}{plt}\PY{o}{.}\PY{n}{legend}\PY{p}{(}\PY{p}{)}
\PY{n}{plt}\PY{o}{.}\PY{n}{xlim}\PY{p}{(}\PY{l+m+mi}{0}\PY{p}{,}\PY{n}{xf}\PY{o}{.}\PY{n}{max}\PY{p}{(}\PY{l+m+mi}{0}\PY{p}{)}\PY{p}{)}
\PY{n}{plt}\PY{o}{.}\PY{n}{ylim}\PY{p}{(}\PY{o}{\PYZhy{}}\PY{l+m+mi}{2}\PY{p}{,}\PY{l+m+mi}{2}\PY{p}{)}
\PY{n}{plt}\PY{o}{.}\PY{n}{xlabel}\PY{p}{(}\PY{l+s+s1}{\PYZsq{}}\PY{l+s+s1}{freq (Hz)}\PY{l+s+s1}{\PYZsq{}}\PY{p}{)}
\PY{n}{plt}\PY{o}{.}\PY{n}{ylabel}\PY{p}{(}\PY{l+s+s1}{\PYZsq{}}\PY{l+s+s1}{amplitude}\PY{l+s+s1}{\PYZsq{}}\PY{p}{)}
\PY{n}{plt}\PY{o}{.}\PY{n}{grid}\PY{p}{(}\PY{p}{)}
\end{Verbatim}
\end{tcolorbox}

    \begin{center}
    \adjustimage{max size={0.9\linewidth}{0.9\paperheight}}{output_13_0.png}
    \end{center}
    { \hspace*{\fill} \\}
    
    \begin{tcolorbox}[breakable, size=fbox, boxrule=1pt, pad at break*=1mm,colback=cellbackground, colframe=cellborder]
\prompt{In}{incolor}{176}{\boxspacing}
\begin{Verbatim}[commandchars=\\\{\}]
\PY{c+c1}{\PYZsh{} Aproximação da onda dente de serra via somatório de harmônicas da série de Fourier}

\PY{n}{ncoeff} \PY{o}{=} \PY{l+m+mi}{10} \PY{c+c1}{\PYZsh{} número de componentes harmônicos (incluindo componente dc, n=0)}

\PY{n}{dente\PYZus{}aprox} \PY{o}{=} \PY{l+m+mi}{0}
\PY{k}{for} \PY{n}{n} \PY{o+ow}{in} \PY{n+nb}{range}\PY{p}{(}\PY{l+m+mi}{0}\PY{p}{,}\PY{n}{ncoeff}\PY{p}{)}\PY{p}{:}
    \PY{k}{if} \PY{n}{n} \PY{o}{!=} \PY{l+m+mi}{0}\PY{p}{:}
        \PY{n}{dente\PYZus{}aprox} \PY{o}{=} \PY{n}{dente\PYZus{}aprox} \PY{o}{+} \PY{n}{dente\PYZus{}an}\PY{p}{[}\PY{n}{xf}\PY{o}{==}\PY{n}{n}\PY{o}{*}\PY{n}{f0}\PY{p}{]}\PY{o}{*}\PY{n}{cos}\PY{p}{(}\PY{l+m+mi}{2}\PY{o}{*}\PY{n}{π}\PY{o}{*}\PY{n}{n}\PY{o}{*}\PY{n}{f0}\PY{o}{*}\PY{n}{t}\PY{p}{)}\PYZbs{}
                                  \PY{o}{+} \PY{n}{dente\PYZus{}bn}\PY{p}{[}\PY{n}{xf}\PY{o}{==}\PY{n}{n}\PY{o}{*}\PY{n}{f0}\PY{p}{]}\PY{o}{*}\PY{n}{sin}\PY{p}{(}\PY{l+m+mi}{2}\PY{o}{*}\PY{n}{π}\PY{o}{*}\PY{n}{n}\PY{o}{*}\PY{n}{f0}\PY{o}{*}\PY{n}{t}\PY{p}{)} \PY{c+c1}{\PYZsh{} soma n\PYZhy{}ésimo componente harmônico}
    \PY{k}{else}\PY{p}{:}
        \PY{n}{dente\PYZus{}aprox} \PY{o}{=} \PY{n}{dente\PYZus{}an}\PY{p}{[}\PY{n}{xf}\PY{o}{==}\PY{n}{n}\PY{o}{*}\PY{n}{f0}\PY{p}{]}\PY{o}{/}\PY{l+m+mi}{2}

\PY{n}{plt}\PY{o}{.}\PY{n}{plot}\PY{p}{(}\PY{n}{t}\PY{p}{,} \PY{n}{dente}\PY{p}{,}\PY{l+s+s1}{\PYZsq{}}\PY{l+s+s1}{k\PYZhy{}\PYZhy{}}\PY{l+s+s1}{\PYZsq{}}\PY{p}{,}\PY{n}{label} \PY{o}{=} \PY{l+s+s1}{\PYZsq{}}\PY{l+s+s1}{original}\PY{l+s+s1}{\PYZsq{}}\PY{p}{)}        
\PY{n}{plt}\PY{o}{.}\PY{n}{plot}\PY{p}{(}\PY{n}{t}\PY{p}{,} \PY{n}{dente\PYZus{}aprox}\PY{p}{,}\PY{l+s+s1}{\PYZsq{}}\PY{l+s+s1}{b}\PY{l+s+s1}{\PYZsq{}}\PY{p}{,}\PY{n}{label} \PY{o}{=} \PY{l+s+s1}{\PYZsq{}}\PY{l+s+s1}{aprox}\PY{l+s+s1}{\PYZsq{}}\PY{p}{)}
\PY{n}{plt}\PY{o}{.}\PY{n}{xlim}\PY{p}{(}\PY{l+m+mi}{0}\PY{p}{,}\PY{n}{t}\PY{o}{.}\PY{n}{max}\PY{p}{(}\PY{l+m+mi}{0}\PY{p}{)}\PY{p}{)}
\PY{n}{plt}\PY{o}{.}\PY{n}{xlabel}\PY{p}{(}\PY{l+s+s1}{\PYZsq{}}\PY{l+s+s1}{tempo (s)}\PY{l+s+s1}{\PYZsq{}}\PY{p}{)}
\PY{n}{plt}\PY{o}{.}\PY{n}{ylabel}\PY{p}{(}\PY{l+s+s1}{\PYZsq{}}\PY{l+s+s1}{amplitude}\PY{l+s+s1}{\PYZsq{}}\PY{p}{)}
\PY{n}{plt}\PY{o}{.}\PY{n}{legend}\PY{p}{(}\PY{p}{)}
\PY{n}{plt}\PY{o}{.}\PY{n}{grid}\PY{p}{(}\PY{p}{)}
\end{Verbatim}
\end{tcolorbox}

    \begin{center}
    \adjustimage{max size={0.9\linewidth}{0.9\paperheight}}{output_14_0.png}
    \end{center}
    { \hspace*{\fill} \\}
    
    \hypertarget{reescrevendo-a-suxe9rie-de-fourier-na-sua-forma-harmuxf4nica}{%
\subsubsection{Reescrevendo a série de Fourier na sua forma
harmônica}\label{reescrevendo-a-suxe9rie-de-fourier-na-sua-forma-harmuxf4nica}}

Podemos expressar a série de Fourier de maneira simplificada utilizando
a sua forma harmônica, utilizando apenas a função cosseno para
representar os componentes harmônicos da função.

\[
f(t)=A_{0} + \sum_{n=1}^{\infty} A_{n} \cdot \cos \left(2\pi t \frac{n}{P}-\theta_{n}\right)
\]

em que

\[A_{0}=\frac{a_{0}}{2},\; A_{n}=\sqrt{a_{n}^{2}+b_{n}^{2}} \; \text { para } n \geq 1, \; \theta_{n}=\arctan \left(\frac{b_{n}}{a_{n}}\right)\]

Esta representação é interessante porque nos permite determinar
diretamente a energia associada a cada componente de frequência do
sinal.

    \begin{tcolorbox}[breakable, size=fbox, boxrule=1pt, pad at break*=1mm,colback=cellbackground, colframe=cellborder]
\prompt{In}{incolor}{180}{\boxspacing}
\begin{Verbatim}[commandchars=\\\{\}]
\PY{c+c1}{\PYZsh{} Aproximação da onda quadrada via somatório de harmônicas da série de Fourier}

\PY{n}{ncoeffs} \PY{o}{=} \PY{l+m+mi}{10} \PY{c+c1}{\PYZsh{} número de componentes harmônicos (incluindo componente dc, n=0)}

\PY{n}{quad\PYZus{}an} \PY{o}{=} \PY{n}{np}\PY{o}{.}\PY{n}{zeros}\PY{p}{(}\PY{p}{(}\PY{n}{ncoeffs}\PY{p}{,} \PY{l+m+mi}{1}\PY{p}{)}\PY{p}{)}
\PY{n}{quad\PYZus{}bn} \PY{o}{=} \PY{n}{np}\PY{o}{.}\PY{n}{zeros}\PY{p}{(}\PY{p}{(}\PY{n}{ncoeffs}\PY{p}{,} \PY{l+m+mi}{1}\PY{p}{)}\PY{p}{)}
\PY{n}{xf}      \PY{o}{=} \PY{n}{np}\PY{o}{.}\PY{n}{zeros}\PY{p}{(}\PY{p}{(}\PY{n}{ncoeffs}\PY{p}{,} \PY{l+m+mi}{1}\PY{p}{)}\PY{p}{)}
\PY{n}{An}      \PY{o}{=} \PY{n}{np}\PY{o}{.}\PY{n}{zeros}\PY{p}{(}\PY{p}{(}\PY{n}{ncoeffs}\PY{p}{,} \PY{l+m+mi}{1}\PY{p}{)}\PY{p}{)}
\PY{n}{θn}      \PY{o}{=} \PY{n}{np}\PY{o}{.}\PY{n}{zeros}\PY{p}{(}\PY{p}{(}\PY{n}{ncoeffs}\PY{p}{,} \PY{l+m+mi}{1}\PY{p}{)}\PY{p}{)}

\PY{k}{for} \PY{n}{n} \PY{o+ow}{in} \PY{n+nb}{range}\PY{p}{(}\PY{l+m+mi}{0}\PY{p}{,}\PY{n}{ncoeffs}\PY{p}{)}\PY{p}{:} \PY{c+c1}{\PYZsh{} calcula coeficientes de Fourier para n=0 até n=ncoeffs\PYZhy{}1}
    \PY{n}{quad\PYZus{}an}\PY{p}{[}\PY{n}{n}\PY{p}{]}\PY{p}{,} \PY{n}{quad\PYZus{}bn}\PY{p}{[}\PY{n}{n}\PY{p}{]} \PY{o}{=} \PY{n}{fourierCoeff}\PY{p}{(}\PY{n}{t}\PY{p}{,} \PY{n}{quad} \PY{p}{,} \PY{l+m+mi}{1}\PY{o}{/}\PY{n}{f0}\PY{p}{,} \PY{n}{n}\PY{p}{)}
    \PY{n}{xf}\PY{p}{[}\PY{n}{n}\PY{p}{]} \PY{o}{=} \PY{n}{n}\PY{o}{*}\PY{n}{f0}
    
\PY{k}{for} \PY{n}{n} \PY{o+ow}{in} \PY{n+nb}{range}\PY{p}{(}\PY{l+m+mi}{0}\PY{p}{,}\PY{n}{ncoeffs}\PY{p}{)}\PY{p}{:}
    \PY{k}{if} \PY{n}{n} \PY{o}{!=} \PY{l+m+mi}{0}\PY{p}{:}
        \PY{n}{An}\PY{p}{[}\PY{n}{n}\PY{p}{]} \PY{o}{=} \PY{n}{sqrt}\PY{p}{(}\PY{n}{quad\PYZus{}an}\PY{p}{[}\PY{n}{n}\PY{p}{]}\PY{o}{*}\PY{o}{*}\PY{l+m+mi}{2} \PY{o}{+} \PY{n}{quad\PYZus{}bn}\PY{p}{[}\PY{n}{n}\PY{p}{]}\PY{o}{*}\PY{o}{*}\PY{l+m+mi}{2}\PY{p}{)}
        \PY{n}{θn}\PY{p}{[}\PY{n}{n}\PY{p}{]} \PY{o}{=} \PY{n}{arctan2}\PY{p}{(}\PY{n}{quad\PYZus{}bn}\PY{p}{[}\PY{n}{n}\PY{p}{]}\PY{p}{,} \PY{n}{quad\PYZus{}an}\PY{p}{[}\PY{n}{n}\PY{p}{]}\PY{p}{)}      \PY{c+c1}{\PYZsh{} calcula arctan(bn/an)                    }
    \PY{k}{else}\PY{p}{:}
        \PY{n}{An}\PY{p}{[}\PY{n}{n}\PY{p}{]} \PY{o}{=} \PY{n}{quad\PYZus{}an}\PY{p}{[}\PY{n}{n}\PY{p}{]}\PY{o}{/}\PY{l+m+mi}{2}        


\PY{n}{plt}\PY{o}{.}\PY{n}{figure}\PY{p}{(}\PY{p}{)}
\PY{p}{(}\PY{n}{markers}\PY{p}{,} \PY{n}{stemlines}\PY{p}{,} \PY{n}{baseline}\PY{p}{)} \PY{o}{=} \PY{n}{plt}\PY{o}{.}\PY{n}{stem}\PY{p}{(}\PY{n}{xf}\PY{p}{,} \PY{n}{An}\PY{p}{,}\PY{l+s+s1}{\PYZsq{}}\PY{l+s+s1}{b}\PY{l+s+s1}{\PYZsq{}}\PY{p}{,} \PY{n}{use\PYZus{}line\PYZus{}collection}\PY{o}{=}\PY{k+kc}{True}\PY{p}{,}\PY{n}{label} \PY{o}{=} \PY{l+s+s1}{\PYZsq{}}\PY{l+s+s1}{coeficientes An}\PY{l+s+s1}{\PYZsq{}}\PY{p}{)}
\PY{n}{plt}\PY{o}{.}\PY{n}{setp}\PY{p}{(}\PY{n}{baseline}\PY{p}{,} \PY{n}{visible}\PY{o}{=}\PY{k+kc}{False}\PY{p}{)}
\PY{n}{plt}\PY{o}{.}\PY{n}{setp}\PY{p}{(}\PY{n}{markers}\PY{p}{,} \PY{n}{marker}\PY{o}{=}\PY{l+s+s1}{\PYZsq{}}\PY{l+s+s1}{o}\PY{l+s+s1}{\PYZsq{}}\PY{p}{,} \PY{n}{markersize}\PY{o}{=}\PY{l+m+mi}{5}\PY{p}{,} \PY{n}{markeredgecolor}\PY{o}{=}\PY{l+s+s2}{\PYZdq{}}\PY{l+s+s2}{b}\PY{l+s+s2}{\PYZdq{}}\PY{p}{,} \PY{n}{markeredgewidth}\PY{o}{=}\PY{l+m+mi}{2}\PY{p}{)}
\PY{n}{plt}\PY{o}{.}\PY{n}{setp}\PY{p}{(}\PY{n}{stemlines}\PY{p}{,} \PY{n}{linestyle}\PY{o}{=}\PY{l+s+s2}{\PYZdq{}}\PY{l+s+s2}{\PYZhy{}\PYZhy{}}\PY{l+s+s2}{\PYZdq{}}\PY{p}{,} \PY{n}{color}\PY{o}{=}\PY{l+s+s2}{\PYZdq{}}\PY{l+s+s2}{b}\PY{l+s+s2}{\PYZdq{}}\PY{p}{,} \PY{n}{linewidth}\PY{o}{=}\PY{l+m+mf}{0.5} \PY{p}{)}

\PY{n}{plt}\PY{o}{.}\PY{n}{title}\PY{p}{(}\PY{l+s+s1}{\PYZsq{}}\PY{l+s+s1}{Onda quadrada}\PY{l+s+s1}{\PYZsq{}}\PY{p}{)}
\PY{n}{plt}\PY{o}{.}\PY{n}{legend}\PY{p}{(}\PY{p}{)}
\PY{n}{plt}\PY{o}{.}\PY{n}{xlim}\PY{p}{(}\PY{l+m+mi}{0}\PY{p}{,}\PY{n}{xf}\PY{o}{.}\PY{n}{max}\PY{p}{(}\PY{l+m+mi}{0}\PY{p}{)}\PY{p}{)}
\PY{n}{plt}\PY{o}{.}\PY{n}{ylim}\PY{p}{(}\PY{o}{\PYZhy{}}\PY{l+m+mi}{2}\PY{p}{,}\PY{l+m+mi}{2}\PY{p}{)}
\PY{n}{plt}\PY{o}{.}\PY{n}{xlabel}\PY{p}{(}\PY{l+s+s1}{\PYZsq{}}\PY{l+s+s1}{freq (Hz)}\PY{l+s+s1}{\PYZsq{}}\PY{p}{)}
\PY{n}{plt}\PY{o}{.}\PY{n}{ylabel}\PY{p}{(}\PY{l+s+s1}{\PYZsq{}}\PY{l+s+s1}{amplitude}\PY{l+s+s1}{\PYZsq{}}\PY{p}{)}
\PY{n}{plt}\PY{o}{.}\PY{n}{grid}\PY{p}{(}\PY{p}{)}       

\PY{n}{plt}\PY{o}{.}\PY{n}{figure}\PY{p}{(}\PY{p}{)}
\PY{p}{(}\PY{n}{markers}\PY{p}{,} \PY{n}{stemlines}\PY{p}{,} \PY{n}{baseline}\PY{p}{)} \PY{o}{=} \PY{n}{plt}\PY{o}{.}\PY{n}{stem}\PY{p}{(}\PY{n}{xf}\PY{p}{,} \PY{n}{θn}\PY{p}{,}\PY{l+s+s1}{\PYZsq{}}\PY{l+s+s1}{b}\PY{l+s+s1}{\PYZsq{}}\PY{p}{,} \PY{n}{use\PYZus{}line\PYZus{}collection}\PY{o}{=}\PY{k+kc}{True}\PY{p}{,}\PY{n}{label} \PY{o}{=} \PY{l+s+s1}{\PYZsq{}}\PY{l+s+s1}{fases θn}\PY{l+s+s1}{\PYZsq{}}\PY{p}{)}
\PY{n}{plt}\PY{o}{.}\PY{n}{setp}\PY{p}{(}\PY{n}{baseline}\PY{p}{,} \PY{n}{visible}\PY{o}{=}\PY{k+kc}{False}\PY{p}{)}
\PY{n}{plt}\PY{o}{.}\PY{n}{setp}\PY{p}{(}\PY{n}{markers}\PY{p}{,} \PY{n}{marker}\PY{o}{=}\PY{l+s+s1}{\PYZsq{}}\PY{l+s+s1}{o}\PY{l+s+s1}{\PYZsq{}}\PY{p}{,} \PY{n}{markersize}\PY{o}{=}\PY{l+m+mi}{5}\PY{p}{,} \PY{n}{markeredgecolor}\PY{o}{=}\PY{l+s+s2}{\PYZdq{}}\PY{l+s+s2}{b}\PY{l+s+s2}{\PYZdq{}}\PY{p}{,} \PY{n}{markeredgewidth}\PY{o}{=}\PY{l+m+mi}{2}\PY{p}{)}
\PY{n}{plt}\PY{o}{.}\PY{n}{setp}\PY{p}{(}\PY{n}{stemlines}\PY{p}{,} \PY{n}{linestyle}\PY{o}{=}\PY{l+s+s2}{\PYZdq{}}\PY{l+s+s2}{\PYZhy{}\PYZhy{}}\PY{l+s+s2}{\PYZdq{}}\PY{p}{,} \PY{n}{color}\PY{o}{=}\PY{l+s+s2}{\PYZdq{}}\PY{l+s+s2}{b}\PY{l+s+s2}{\PYZdq{}}\PY{p}{,} \PY{n}{linewidth}\PY{o}{=}\PY{l+m+mf}{0.5} \PY{p}{)}

\PY{n}{plt}\PY{o}{.}\PY{n}{title}\PY{p}{(}\PY{l+s+s1}{\PYZsq{}}\PY{l+s+s1}{Onda quadrada}\PY{l+s+s1}{\PYZsq{}}\PY{p}{)}
\PY{n}{plt}\PY{o}{.}\PY{n}{legend}\PY{p}{(}\PY{p}{)}
\PY{n}{plt}\PY{o}{.}\PY{n}{xlim}\PY{p}{(}\PY{l+m+mi}{0}\PY{p}{,}\PY{n}{xf}\PY{o}{.}\PY{n}{max}\PY{p}{(}\PY{l+m+mi}{0}\PY{p}{)}\PY{p}{)}
\PY{n}{plt}\PY{o}{.}\PY{n}{ylim}\PY{p}{(}\PY{o}{\PYZhy{}}\PY{l+m+mi}{2}\PY{o}{*}\PY{n}{π} \PY{p}{,}\PY{l+m+mi}{2}\PY{o}{*}\PY{n}{π}\PY{p}{)}
\PY{n}{plt}\PY{o}{.}\PY{n}{xlabel}\PY{p}{(}\PY{l+s+s1}{\PYZsq{}}\PY{l+s+s1}{freq (Hz)}\PY{l+s+s1}{\PYZsq{}}\PY{p}{)}
\PY{n}{plt}\PY{o}{.}\PY{n}{ylabel}\PY{p}{(}\PY{l+s+s1}{\PYZsq{}}\PY{l+s+s1}{fase [rad]}\PY{l+s+s1}{\PYZsq{}}\PY{p}{)}
\PY{n}{plt}\PY{o}{.}\PY{n}{grid}\PY{p}{(}\PY{p}{)}


\PY{n}{quad\PYZus{}aprox} \PY{o}{=} \PY{l+m+mi}{0}

\PY{k}{for} \PY{n}{n} \PY{o+ow}{in} \PY{n+nb}{range}\PY{p}{(}\PY{l+m+mi}{0}\PY{p}{,}\PY{n}{ncoeffs}\PY{p}{)}\PY{p}{:}
    \PY{k}{if} \PY{n}{n} \PY{o}{!=} \PY{l+m+mi}{0}\PY{p}{:}
        \PY{n}{quad\PYZus{}aprox} \PY{o}{=} \PY{n}{quad\PYZus{}aprox} \PY{o}{+} \PY{n}{An}\PY{p}{[}\PY{n}{n}\PY{p}{]}\PY{o}{*}\PY{n}{cos}\PY{p}{(}\PY{l+m+mi}{2}\PY{o}{*}\PY{n}{π}\PY{o}{*}\PY{n}{n}\PY{o}{*}\PY{n}{f0}\PY{o}{*}\PY{n}{t} \PY{o}{\PYZhy{}} \PY{n}{θn}\PY{p}{[}\PY{n}{n}\PY{p}{]}\PY{p}{)} \PY{c+c1}{\PYZsh{} soma n\PYZhy{}ésimo componente harmônico}
                                 
    \PY{k}{else}\PY{p}{:}
        \PY{n}{quad\PYZus{}aprox} \PY{o}{=} \PY{n}{An}\PY{p}{[}\PY{n}{n}\PY{p}{]}
        
        
\PY{n}{plt}\PY{o}{.}\PY{n}{figure}\PY{p}{(}\PY{p}{)}
\PY{n}{plt}\PY{o}{.}\PY{n}{plot}\PY{p}{(}\PY{n}{t}\PY{p}{,} \PY{n}{quad}\PY{p}{,}\PY{l+s+s1}{\PYZsq{}}\PY{l+s+s1}{k\PYZhy{}\PYZhy{}}\PY{l+s+s1}{\PYZsq{}}\PY{p}{,}\PY{n}{label} \PY{o}{=} \PY{l+s+s1}{\PYZsq{}}\PY{l+s+s1}{original}\PY{l+s+s1}{\PYZsq{}}\PY{p}{)}        
\PY{n}{plt}\PY{o}{.}\PY{n}{plot}\PY{p}{(}\PY{n}{t}\PY{p}{,}\PY{n}{quad\PYZus{}aprox}\PY{p}{,}\PY{l+s+s1}{\PYZsq{}}\PY{l+s+s1}{b}\PY{l+s+s1}{\PYZsq{}}\PY{p}{,}\PY{n}{label} \PY{o}{=} \PY{l+s+s1}{\PYZsq{}}\PY{l+s+s1}{aprox}\PY{l+s+s1}{\PYZsq{}}\PY{p}{)}
\PY{n}{plt}\PY{o}{.}\PY{n}{xlim}\PY{p}{(}\PY{l+m+mi}{0}\PY{p}{,}\PY{n}{t}\PY{o}{.}\PY{n}{max}\PY{p}{(}\PY{l+m+mi}{0}\PY{p}{)}\PY{p}{)}
\PY{n}{plt}\PY{o}{.}\PY{n}{xlabel}\PY{p}{(}\PY{l+s+s1}{\PYZsq{}}\PY{l+s+s1}{tempo (s)}\PY{l+s+s1}{\PYZsq{}}\PY{p}{)}
\PY{n}{plt}\PY{o}{.}\PY{n}{ylabel}\PY{p}{(}\PY{l+s+s1}{\PYZsq{}}\PY{l+s+s1}{amplitude}\PY{l+s+s1}{\PYZsq{}}\PY{p}{)}
\PY{n}{plt}\PY{o}{.}\PY{n}{legend}\PY{p}{(}\PY{p}{)}
\PY{n}{plt}\PY{o}{.}\PY{n}{grid}\PY{p}{(}\PY{p}{)}
\end{Verbatim}
\end{tcolorbox}

    \begin{center}
    \adjustimage{max size={0.9\linewidth}{0.9\paperheight}}{output_16_0.png}
    \end{center}
    { \hspace*{\fill} \\}
    
    \begin{center}
    \adjustimage{max size={0.9\linewidth}{0.9\paperheight}}{output_16_1.png}
    \end{center}
    { \hspace*{\fill} \\}
    
    \begin{center}
    \adjustimage{max size={0.9\linewidth}{0.9\paperheight}}{output_16_2.png}
    \end{center}
    { \hspace*{\fill} \\}
    
    \hypertarget{distoruxe7uxe3o}{%
\subsection{Distorção}\label{distoruxe7uxe3o}}

O conceito de distorção aparece no contexto das disciplinas relacionadas
à engenharia elétrica, particulamente nas áreas de processamento de
sinais e comunicações. Diz-se que um sinal sofreu distorção quando o
mesmo teve a sua forma original modificada, excetuando-se os casos em
que tal modificação corresponde a uma mudança de escala de amplitude, ou
a um atraso no tempo, que obviamente não alteram a forma de nenhum
sinal. Distorções podem ser classificadas em dois tipos básicos:

\textbf{Distorções lineares}: causadas por operações lineares sofridas
pelo sinal.

\textbf{Distorções não-lineares}: causadas por operações não-lineares
sofridas pelo sinal.

\hypertarget{quando-um-sistema-linear-e-invariante-no-tempo-lit-nuxe3o-causaruxe1-distoruxe7uxe3o}{%
\subsubsection{Quando um sistema linear e invariante no tempo (LIT) não
causará
distorção?}\label{quando-um-sistema-linear-e-invariante-no-tempo-lit-nuxe3o-causaruxe1-distoruxe7uxe3o}}

As duas operações lineares que não distorcem um sinal estão resumidas em

 
\begin{equation}
y(t)=Gx(t-\tau).
\label{eq1} \tag{1}
\end{equation}


Podemos considerar \(y(t)\) \(\eqref{eq1}\) como sendo a saída de um
sistema LIT, cuja resposta ao impulso é \(h(t)=G\delta(t-\tau)\), quando
o sinal \(x(t)\) é aplicado na entrada. Analisando \(\eqref{eq1}\) no
domínio da frequência, temos

 
\begin{align}
Y(f)&=Ge^{-j2\pi f \tau}X(f)\\ \\
Y(f)&=H(f)X(f), \text { com } H(f) = Ge^{-j2\pi f \tau}.
\label{eq2} \tag{2}
\end{align}


Note que a resposta em amplitude de \(H(f)\) é constante
\(\left(|H(f)|=G\right)\) e a resposta em fase
\(\left(\angle H(f)=-2\pi f \tau\right)\) é uma função linear de \(f\).
Portanto, a menos que o sistema LTI possua resposta em frequência na
forma expressa em \(\eqref{eq2}\), \(y(t)\) será uma versão distorcida
de \(x(t)\).

\hypertarget{verificauxe7uxe3o-para-o-caso-de-sinais-periuxf3dicos}{%
\paragraph{Verificação para o caso de sinais
periódicos}\label{verificauxe7uxe3o-para-o-caso-de-sinais-periuxf3dicos}}

Considerando que o sinal \(x(t)\) é periódico e admite representação em
termos de uma série de Fourier, temos

\[
x(t)=A_{0} + \sum_{n=1}^{\infty} A_{n}\cos \left(2\pi n f_0 t-\theta_{n}\right)
\]

logo, de \(\eqref{eq1}\) temos que

\[
\begin{align}
y(t)&=GA_{0} + \sum_{n=1}^{\infty} GA_{n}\cos \left[2\pi n f_0 (t-\tau) -\theta_{n}\right]\\
y(t)&=GA_{0} + \sum_{n=1}^{\infty} GA_{n}\cos \left[2\pi n f_0 t -2\pi n f_0 \tau  -\theta_{n}\right]\\
y(t)&=GA_{0} + \sum_{n=1}^{\infty} GA_{n}\cos \left[2\pi n f_0 t -\theta_{n} + \Delta \theta\right]
\end{align}
\]

Desse modo, perceba que para que o sinal \(x(t)\) não sofra distorção,
todos os seus componentes de frequência devem sofrer o mesmo ganho \(G\)
(ou atenuação, se \(|G|<1\)) e o desvio de fase \(\Delta \theta\) gerado
pelo atraso \(\tau\) deve ser linear com relação à frequência
\(\Delta \theta = (-2\pi f_0 \tau)n\), o que está de acordo com
\(\eqref{eq2}\)

    \begin{tcolorbox}[breakable, size=fbox, boxrule=1pt, pad at break*=1mm,colback=cellbackground, colframe=cellborder]
\prompt{In}{incolor}{ }{\boxspacing}
\begin{Verbatim}[commandchars=\\\{\}]

\end{Verbatim}
\end{tcolorbox}


    % Add a bibliography block to the postdoc
    
    
    
\end{document}
